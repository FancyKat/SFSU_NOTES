\documentclass[12pt]{article}
\usepackage{amsmath}
\usepackage{graphicx}
\usepackage{url}
\usepackage{hyperref}

\title{Exploring the Fibonacci Sequence's Application in Network Load Balancing}
\author{Your Name}
\date{\today}

\begin{document}

\maketitle

\begin{abstract}
This paper explores the application of the Fibonacci sequence in the context of network load balancing, specifically through the Fibonacci Multipath Load Balancing (FMLB) protocol. We delve into how the inherent properties of the Fibonacci sequence and the golden ratio provide an innovative approach to managing network congestion, enhancing packet delivery ratios, and optimizing end-to-end delay in Mobile Ad Hoc Networks (MANETs).
\end{abstract}

\section{Introduction}
The Fibonacci sequence, a series of numbers where each number is the sum of the two preceding ones, manifests across various facets of nature and mathematics. This paper examines a novel application of this sequence in network load balancing, particularly within the realm of MANETs, through the FMLB protocol.

\section{Background}
\subsection{The Fibonacci Sequence and Golden Ratio}
\begin{itemize}
    \item Definition and properties of the Fibonacci sequence.
    \item The golden ratio and its relationship to the Fibonacci sequence.
\end{itemize}

\subsection{Network Load Balancing and MANETs}
\begin{itemize}
    \item Overview of network load balancing: objectives and challenges.
    \item Introduction to Mobile Ad Hoc Networks (MANETs) and their characteristics.
\end{itemize}

\section{FMLB Protocol}
\subsection{Overview}
Discussion on how FMLB leverages the Fibonacci sequence for load balancing in MANETs.

\subsection{Operation}
Detailed explanation of FMLB's operation, including path discovery and packet distribution according to Fibonacci weights.

\subsection{Benefits}
Analysis of how FMLB improves upon traditional load balancing approaches in terms of packet delivery ratio and delay minimization.

\section{Case Study and Results}
Presentation of a case study or simulation results demonstrating FMLB's performance in comparison to other protocols.

\section{Conclusion}
Summarization of FMLB's advantages in network load balancing, its implications for future network design, and potential areas for further research.

\bibliographystyle{IEEEtran}
\bibliography{references}

\end{document}

