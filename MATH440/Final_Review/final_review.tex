\documentclass[12pt]{article}

% Packages
\usepackage[margin=1in]{geometry}
\usepackage{fancyhdr}
\usepackage{hyperref}
\usepackage{datetime} % For formatting the current date
\usepackage{parskip} % Adds space between paragraphs and removes paragraph indents
\usepackage{amsmath}
\usepackage{amssymb}

% Hyperlink settings
\hypersetup{
    colorlinks=true, % Colored links instead of boxes
    urlcolor=blue,   % Blue color for external links
}

% Custom date format
\newdateformat{monthyeardate}{%
\monthname[\THEMONTH] \THEYEAR}

% Header settings
\pagestyle{fancy}
\fancyhf{} % Clear all header and footer fields
\lhead{MATH 440 \\ 2:00 PM - 3:20 PM} % Left header with class number and time
\rhead{Marty Martin \\ \monthyeardate\today} % Right header with your name and the date
\renewcommand{\headrulewidth}{0.4pt} % Header underlining
\setlength{\headheight}{54pt} % Ensure there's enough space for two lines in the header

\begin{document}

\begin{center}
  \Large \textbf{Practice for the exam}
\end{center}

\subsection*{Question 1}
Suppose the random variables \(X\) and \(Y\) are jointly distributed according to the pdf:
\[ f_{XY}(x,y) = 8xy, \quad 0 < y < x < 1 \]
\begin{itemize}
    \item[(a)] Find \( P(X < 2Y) \)
    \item[(b)] Find \( P(Y < \frac{1}{4} |  x = \frac{1}{2}) \)
\end{itemize}

% Part (a) explanation and calculation
\subsubsection*{Part (a) Find \( P(X < 2Y) \)}
% General formula for probability calculation using a joint PDF
\textbf{General Formula:}
% Introduce the general integral formula for calculating probability
For any joint PDF \( f_{XY}(x, y) \), the probability \( P(g(X, Y)) \) for some condition \( g \) is given by:
\[ P(g(X, Y)) = \int \int_{g(x, y)} f_{XY}(x, y) \, dx \, dy \]

% Specific Calculation:
This probability calculation involves integrating the joint PDF over the region defined by \( X < 2Y \) and within the given bounds of \( 0 < y < x < 1 \).
\[ 
P(X < 2Y) = \int_0^1 \int_0^{X/2} 8xy \, dy \, dx 
\]
This integral setup reflects the condition \( X < 2Y \) within the area bounded by \( 0 < y < x < 1 \).

% Integration Steps:
Calculate the inner integral over \( y \):
\[
\int_0^{X/2} 8xy \, dy = 8x \left[ \frac{y^2}{2} \right]_0^{X/2} = 8x \left[ \frac{(X/2)^2}{2} \right] = X^3
\]
Now integrate with respect to \( x \):
\[
\int_0^1 X^3 \, dx = \left[ \frac{X^4}{4} \right]_0^1 = \frac{1}{4}
\]

Thus, \( P(X < 2Y) = \frac{1}{4} \).


\newpage
% Part (b) explanation and calculation
\subsubsection*{Part (b) Find \( P(Y < \frac{1}{4} |  x = \frac{1}{2}) \)}
% General formula for conditional probability:
For any joint PDF \( f_{XY}(x, y) \), the conditional probability \( P(A | B) \) is given by:
\[ P(A | B) = \frac{\int \int_{A \cap B} f_{XY}(x, y) \, dx \, dy}{\int \int_B f_{XY}(x, y) \, dx \, dy} \]

% Specific Calculation:
Given \( X = \frac{1}{2} \), we need to find the conditional probability \( P(Y < \frac{1}{4} | X = \frac{1}{2}) \). This involves determining the conditional PDF \( f_{Y|X}(y|x) \) and integrating it over the desired range of \( Y \).

% Determine the marginal density of X, f_X(x):
The marginal density \( f_X(x) \) is found by integrating out \( Y \) from the joint PDF:
\[ f_X(x) = \int_0^x 8xy \, dy = 8x \left[ \frac{y^2}{2} \right]_0^x = 4x^3 \]
At \( x = \frac{1}{2} \), the marginal density \( f_X\left(\frac{1}{2}\right) \) is:
\[ f_X\left(\frac{1}{2}\right) = 4 \left(\frac{1}{2}\right)^3 = \frac{1}{2} \]

% Calculate the conditional PDF f_{Y|X}(y|x):
The conditional PDF \( f_{Y|X}(y|\frac{1}{2}) \) is:
\[ f_{Y|X}(y|\frac{1}{2}) = \frac{8 \cdot \frac{1}{2} \cdot y}{\frac{1}{2}} = 8y \]

% Compute the conditional probability P(Y < 1/4 | X = 1/2):
\[ P(Y < \frac{1}{4} | X = \frac{1}{2}) = \int_0^{1/4} 8y \, dy \]
Calculate the integral:
\[ \int_0^{1/4} 8y \, dy = \left[ 4y^2 \right]_0^{1/4} = 4 \left(\frac{1}{16}\right) = \frac{1}{4} \]

% Conclusion:
Therefore, \( P(Y < \frac{1}{4} | X = \frac{1}{2}) = \frac{1}{4} \).

\newpage
\subsection*{Question 2}
A random variable has moment generating function \( M_X(t) = \left(\frac{2+e^t}{3}\right)^9 \). Find \( \text{Var}(X) \).

\newpage
\subsection*{Question 3}
The driver of a truck loaded with 900 boxes of books will be fined if the total weight of the boxes exceeds 36450 pounds. If the distribution of the weight of a box has a mean of 40 pounds and a variance of 36, find the approximate probability that the driver will be fined.

\newpage
\subsection*{Question 4}
Suppose that the number of calls per hour to an answering service follows a Poisson distribution with rate \( \lambda = 4 \).
\begin{itemize}
  \item[(a)] What is the probability that fewer than 2 calls came in the first hour?
  \item[(b)] What is the probability that there will be no calls in the next two hours?
\end{itemize}

\newpage
\subsection*{Question 5}
Using moment generating functions (MGFs), show that if:
\[ X \sim N(\mu_1, \sigma_1^2) \quad \text{and} \quad Y \sim N(\mu_2, \sigma_2^2), \]
then the expectation and variance of \( X+Y \) are given by:
\[ E(X+Y) = \mu_1 + \mu_2 \quad \text{and} \quad \text{Var}(X+Y) = \sigma_1^2 + \sigma_2^2, \]
and that:
\[ X + Y \sim N(\mu_1 + \mu_2, \sigma_1^2 + \sigma_2^2). \]
\textit{Note:} The moment generating function of \( X \), if \( X \) is normally distributed, is given by:
\[ M_X(t) = e^{t\mu + \frac{t^2\sigma^2}{2}}. \]


\end{document}
