% Axiom 1.1
\unnumberedsection{Axiom 1.1: Properties of Integers}
If $m$, $n$, and $p$ are integers, then:
\begin{enumerate}[label=(\roman*)]
    \item $m+n = n+m$ \hfill (\textbf{commutativity of addition})
    \item $(m+n)+ p = m+(n+ p)$ \hfill (\textbf{associativity of addition})
    \item $m\cdot (n+m) = m\cdot n+m\cdot p$ \hfill (\textbf{distributivity})
    \item $m\cdot n = n\cdot m$ \hfill (\textbf{commutativity of multiplication})
    \item $(m\cdot n) \cdot p = m\cdot (n\cdot p)$ \hfill (\textbf{associativity of multiplication})
\end{enumerate}

% Axiom 1.2
\unnumberedsection{Axiom 1.2: Identity Element for Addition}
There exists an integer $0$ such that whenever $m \in \mathbb{Z}$, $m+0 = m$ (\textbf{identity element for addition}).

% Axiom 1.3
\unnumberedsection{Axiom 1.3: Identity Element for Multiplication}
There exists an integer $1$ such that $1 \neq 0$ and whenever $m \in \mathbb{Z}$, $m\cdot 1 = m$ (\textbf{identity element for multiplication}).

% Axiom 1.4
\unnumberedsection{Axiom 1.4: Additive Inverse}
For each $m \in \mathbb{Z}$, there exists an integer, denoted by $-m$, such that $m+(-m) = 0$ (\textbf{additive inverse}).

% Axiom 1.5
\unnumberedsection{Axiom 1.5: Cancellation}
Let $m$, $n$, and $p$ be integers. If $m\cdot n = m\cdot p$ and $m \neq 0$, then $n = p$ (\textbf{cancellation}).

\section*{Proof Example}
\begin{proof}
  If \(m\) is an integer and \(m \cdot 0 = 0\), then \(m = m\).
  \begin{itemize}
    \item Consider an integer \(m\).
    \item Multiplying by \(0\) gives \(m \cdot 0 = 0\).
    \item Since \(m \cdot 0 = 0\), by the property of zero in multiplication, we have \(m = m\).
    \item Thus, the statement is proven.\qedhere
  \end{itemize}
\end{proof}
