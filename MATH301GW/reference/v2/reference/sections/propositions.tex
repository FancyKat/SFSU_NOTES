This chapter outlines various propositions derived from the axioms of integers and their proofs.

\section*{Properties of Integers}

\subsection*{Proposition 1.6 (Distributive Property)}
For any integers \(m, n,\) and \(p\), the distributive property holds: \((m + n) \cdot p = m \cdot p + n \cdot p\).

\begin{proof}
Consider integers \(m, n,\) and \(p\). By the distributive axiom of integers, we have:
\begin{align*}
    (m + n) \cdot p &= m \cdot p + n \cdot p \\
    \text{Thus, the proposition is proven.} \qedhere
\end{align*}
\end{proof}

% More propositions follow...

\subsection*{Proposition 1.7 (Identity Property)}
For any integer \(m\), the identity property of addition and multiplication holds: 
\begin{itemize}
    \item Adding zero to \(m\) does not change it: \(0 + m = m\).
    \item Multiplying \(m\) by one does not change it: \(1 \cdot m = m\).
\end{itemize}

\subsection*{Proposition 1.8 (Additive Inverse)}
For any integer \(m\), the additive inverse property holds: \((-m) + m = 0\).

\subsection*{Proposition 1.9 (Cancellation Law)}
Let \(m, n,\) and \(p\) be integers. If \(m + n = m + p\), then \(n = p\).

\begin{proof}
Suppose \(m, n,\) and \(p\) are integers such that \(m + n = m + p\). By subtracting \(m\) from both sides of the equation, we cancel \(m\) and conclude that \(n = p\). Thus, the cancellation law is validated.
\end{proof}

% Additional propositions and their proofs can be added following the same format.
