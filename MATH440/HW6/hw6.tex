\documentclass[12pt]{article}

% Packages
\usepackage[margin=1in]{geometry}
\usepackage{fancyhdr}
\usepackage{hyperref}
\usepackage{datetime} % For formatting the current date
\usepackage{parskip} % Adds space between paragraphs and removes paragraph indents
\usepackage{amsmath}
\usepackage{amssymb}

% Hyperlink settings
\hypersetup{
    colorlinks=true, % Colored links instead of boxes
    urlcolor=blue,   % Blue color for external links
}

% Custom date format
\newdateformat{monthyeardate}{%
\monthname[\THEMONTH] \THEYEAR}

% Header settings
\pagestyle{fancy}
\fancyhf{} % Clear all header and footer fields
\lhead{MATH 440 \\ 2:00 PM - 3:20 PM} % Left header with class number and time
\rhead{Marty Martin \\ \monthyeardate\today} % Right header with your name and the date
\renewcommand{\headrulewidth}{0.4pt} % Header underlining
\setlength{\headheight}{54pt} % Ensure there's enough space for two lines in the header

\begin{document}

\begin{center}
  \Large \textbf{Homework 8}
\end{center}

\section*{Problem 3.4.6}
\subsection*{Question}
Let \( n \) be a positive integer. Show that 
\[ f(y) = (n+2)(n+1)y^n(1-y), \quad 0 \leq y \leq 1, \]
is pdf.

\subsection*{Solution}
% To show that f(y) is a pdf, we need to verify two conditions:
% 1. f(y) \geq 0 for all y in the range [0, 1].
% 2. The integral of f(y) from 0 to 1 equals 1.

\subsubsection*{Non-negativity}
% Since y^n \geq 0 and (1-y) \geq 0 for 0 \leq y \leq 1, and the constants (n+2) and (n+1) are positive,
% f(y) is non-negative for all y in [0, 1].
\[
f(y) = (n+2)(n+1)y^n(1-y) \geq 0 \text{ for } 0 \leq y \leq 1
\]

\subsubsection*{Total Integral Equals One}
% We need to calculate the integral of f(y) over [0, 1] and show that it equals 1.
% The integral of y^n(1-y) over [0, 1] is a standard result from calculus related to the beta function.
% Specifically, the integral of y^a(1-y)^b over [0,1] is given by the beta function B(a+1, b+1),
% which simplifies using the gamma function to \frac{\Gamma(a+1)\Gamma(b+1)}{\Gamma(a+b+2)}.
\[
\int_{0}^{1} y^n(1-y) \, dy = \frac{\Gamma(n+1)\Gamma(2)}{\Gamma(n+3)} = \frac{n! \cdot 1}{(n+2)!} = \frac{1}{(n+2)(n+1)}
\]
% Multiplying by the constant (n+2)(n+1) to find the integral of f(y):
\[
\int_{0}^{1} f(y) \, dy = (n+2)(n+1) \int_{0}^{1} y^n(1-y) \, dy = (n+2)(n+1) \cdot \frac{1}{(n+2)(n+1)} = 1
\]
% Therefore, the integral of f(y) over its range is 1, confirming that f(y) is a pdf.

\newpage
\section*{Problem 3.4.9}
\subsection*{Question}
If the PDF for \( Y \) is given by
\[ 
f_Y(y) = 
\begin{cases} 
1 - |y| & \text{if } |y| \leq 1 \\
0 & \text{if } |y| > 1 
\end{cases}
\]
find and graph \( F_Y(y) \).

\subsection*{Solution}
% Calculating F_Y(y) based on the defined regions:
\subsubsection*{For \( y < -1 \)}
\[
F_Y(y) = 0
\]
\subsubsection*{For \( -1 \leq y \leq 1 \)}
\[
F_Y(y) = \int_{-1}^y (1 - |t|) \, dt
\]
% Breaking down the integral based on the value of y:
\[
F_Y(y) = 
\begin{cases} 
\int_{-1}^y (1 + t) \, dt & \text{if } -1 \leq y < 0 \\
\int_{-1}^0 (1 + t) \, dt + \int_0^y (1 - t) \, dt & \text{if } 0 \leq y \leq 1 
\end{cases}
\]
\subsubsection*{For \( y > 1 \)}
\[
F_Y(y) = 1
\]

\newpage
\section*{Problem 3.4.14}
\subsection*{Question}
In a certain country, the distribution of a family's disposable income, \( y \), is described by the pdf \( f_Y(y) = ye^{-y} \) for \( y \geq 0 \). Find \( F_Y(y) \).

\subsection*{Solution}
% Checking if the given function is a valid PDF
\subsubsection*{Verification of PDF}
\[
\text{Non-negativity: } f_Y(y) = ye^{-y} \geq 0 \text{ for } y \geq 0
\]
\[
\text{Integral equals 1: } \int_{0}^{\infty} ye^{-y} \, dy = 1
\]
\subsubsection*{Calculation of CDF}
\[
F_Y(y) = \int_{0}^{y} te^{-t} \, dt
\]
% Using integration by parts
\[
F_Y(y) = [-(t+1)e^{-t}]_{0}^{y} = -(y+1)e^{-y} + 1
\]
\[
F_Y(y) = 1 - (y+1)e^{-y}, \quad y \geq 0
\]


\end{document}
