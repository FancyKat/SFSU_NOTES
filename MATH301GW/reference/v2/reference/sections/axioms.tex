\section*{Axioms of Integers}

The axioms of integers describe the basic properties that define the structure of the set of integers (\(\mathbb{Z}\)).

\subsection*{Axiom 1.1 (Commutativity and Associativity)}
\begin{itemize}
  \item For any integers \(m, n\), the operation of addition is commutative: \(m + n = n + m\).
  \item For any integers \(m, n, p\), the operation of addition is associative: \((m + n) + p = m + (n + p)\).
  \item For any integers \(m, n, p\), the distributive property connects the operations of multiplication and addition: \(m \cdot (n + p) = m \cdot n + m \cdot p\).
  \item For any integers \(m, n\), the operation of multiplication is commutative: \(m \cdot n = n \cdot m\).
  \item For any integers \(m, n, p\), the operation of multiplication is associative: \((m \cdot n) \cdot p = m \cdot (n \cdot p)\).
\end{itemize}

\subsection*{Axiom 1.2 (Identity Elements)}
\begin{itemize}
  \item There exists an integer \(0\) such that for any integer \(m\), adding \(0\) to \(m\) leaves it unchanged: \(m + 0 = m\).
  \item There exists an integer \(1\) (\(1 \neq 0\)) such that for any integer \(m\), multiplying \(m\) by \(1\) leaves it unchanged: \(m \cdot 1 = m\).
\end{itemize}

\subsection*{Axiom 1.3 (Additive Inverse)}
For each integer \(m\), there exists an integer denoted by \(-m\) such that their sum is \(0\): \(m + (-m) = 0\).

\subsection*{Axiom 1.4 (Cancellation Law)}
For any integers \(m, n, p\), if \(m \neq 0\) and \(m \cdot n = m \cdot p\), then \(n = p\).


\section*{Proof Example}
\begin{proof}
  If \(m\) is an integer and \(m \cdot 0 = 0\), then \(m = m\).
  \begin{itemize}
    \item Consider an integer \(m\).
    \item Multiplying by \(0\) gives \(m \cdot 0 = 0\).
    \item Since \(m \cdot 0 = 0\), by the property of zero in multiplication, we have \(m = m\).
    \item Thus, the statement is proven.\qedhere
  \end{itemize}
\end{proof}
