\documentclass[12pt]{article}

% Packages
\usepackage[margin=1in]{geometry}
\usepackage{fancyhdr}
\usepackage{hyperref}
\usepackage{datetime} % For formatting the current date
\usepackage{parskip} % Adds space between paragraphs and removes paragraph indents
\usepackage{amsmath}

% Hyperlink settings
\hypersetup{
    colorlinks=true, % Colored links instead of boxes
    urlcolor=blue,   % Blue color for external links
}

% Custom date format
\newdateformat{monthyeardate}{%
\monthname[\THEMONTH] \THEYEAR}

% Header settings
\pagestyle{fancy}
\fancyhf{} % Clear all header and footer fields
\lhead{MATH325 \\ 2:00 PM - 3:40 PM} % Left header with class number and time
\rhead{Marty Martin \\ \monthyeardate\today} % Right header with your name and the date
\renewcommand{\headrulewidth}{0.4pt} % Header underlining
\setlength{\headheight}{54pt} % Ensure there's enough space for two lines in the header

\begin{document}


\section*{5.4.1 Find the least squares solution to the linear system $Ax = b$}

\begin{enumerate}
  \item[(a)]
    Given:
    \[
      A = \begin{bmatrix} 1 \\ 2 \\ 1 \end{bmatrix}, \quad b = \begin{bmatrix} 1 \\ 1 \\ 0 \end{bmatrix}
    \]

    % The matrix A is a 3x1 column vector, and b is also a 3x1 column vector.

    We need to find the least squares solution to the equation \( Ax = b \). The normal equation for this problem is:
    \[
      A^T A x = A^T b
    \]
    % The normal equation is derived by multiplying both sides of Ax = b by A^T.

    First, calculate \( A^T A \):
    \[
      A^T = \begin{bmatrix} 1 & 2 & 1 \end{bmatrix}
    \]
    % A^T is the transpose of A, converting A from a 3x1 matrix to a 1x3 matrix.

    \[
      A^T A = \begin{bmatrix} 1 & 2 & 1 \end{bmatrix} \begin{bmatrix} 1 \\ 2 \\ 1 \end{bmatrix} = \begin{bmatrix} 6 \end{bmatrix}
    \]
    % A^T A is the product of A^T and A, resulting in a 1x1 matrix, which is a scalar value here.

    Then, calculate \( A^T b \):
    \[
      A^T b = \begin{bmatrix} 1 & 2 & 1 \end{bmatrix} \begin{bmatrix} 1 \\ 1 \\ 0 \end{bmatrix} = \begin{bmatrix} 3 \end{bmatrix}
    \]
    % A^T b is the product of A^T and b, resulting in a 1x1 matrix (scalar).

    Now, solve the normal equation:
    \[
      6x = 3
    \]
    % The normal equation simplifies to a scalar equation because A^T A and A^T b are scalars.

    \[
      x = \frac{3}{6} = 0.5
    \]
    % Solving the scalar equation for x.

    Thus, the least squares solution \( x \) is:
    \[
      x = 0.5
    \]
    % The solution to the normal equation provides the value of x that minimizes the sum of the squares of the residuals in the equation Ax = b.

    \pagebreak

  \item[(b)]
    Given:
    \[
      A = \begin{bmatrix} 1 & 0 \\ 2 & -1 \\ 3 & 5 \end{bmatrix}, \quad b = \begin{bmatrix} 3 \\ 7 \\ 7 \end{bmatrix}
    \]

    We need to find the least squares solution to the equation \( Ax = b \). The normal equation for this problem is:
    \[
      A^T A x = A^T b
    \]
    % The normal equation is derived by multiplying both sides of Ax = b by A^T.

    First, calculate \( A^T A \):
    \[
      A^T = \begin{bmatrix} 1 & 2 & 3 \\ 0 & -1 & 5 \end{bmatrix}
    \]
    % A^T is the transpose of A, converting A from a 3x2 matrix to a 2x3 matrix.

    \[
      A^T A = \begin{bmatrix} 1 & 2 & 3 \\ 0 & -1 & 5 \end{bmatrix} \begin{bmatrix} 1 & 0 \\ 2 & -1 \\ 3 & 5 \end{bmatrix} = \begin{bmatrix} 14 & 13 \\ 13 & 26 \end{bmatrix}
    \]
    % A^T A is the product of A^T and A, resulting in a 2x2 matrix.

    Then, calculate \( A^T b \):
    \[
      A^T b = \begin{bmatrix} 1 & 2 & 3 \\ 0 & -1 & 5 \end{bmatrix} \begin{bmatrix} 3 \\ 7 \\ 7 \end{bmatrix} = \begin{bmatrix} 38 \\ 28 \end{bmatrix}
    \]
    % A^T b is the product of A^T and b, resulting in a 2x1 matrix.

    Now, solve the normal equation:
    \[
      \begin{bmatrix} 14 & 13 \\ 13 & 26 \end{bmatrix} \begin{bmatrix} x_1 \\ x_2 \end{bmatrix} = \begin{bmatrix} 38 \\ 28 \end{bmatrix}
    \]
    % Solving this system of linear equations to find x_1 and x_2.

    Thus, the least squares solution \( x \) is:
    \[
      x = \begin{bmatrix} 3.2 \\ -0.52307692 \end{bmatrix}
    \]
    % The solution provides the values for x_1 and x_2 that minimize the sum of the squares of the residuals in the equation Ax = b.

    \pagebreak

  \item[(c)]
    Given:
    \[
      A = \begin{bmatrix} 2 & 1 & -1 \\ 1 & -2 & 0 \\ 3 & -1 & 1 \end{bmatrix}, \quad b = \begin{bmatrix} 1 \\ 0 \\ 1 \end{bmatrix}
    \]
    % The matrix A is a 3x3 matrix, and b is a 3x1 column vector.

    We need to find the least squares solution to the equation \( Ax = b \). The normal equation for this problem is:
    \[
      A^T A x = A^T b
    \]
    % The normal equation is derived by multiplying both sides of Ax = b by A^T.

    First, calculate \( A^T A \):
    \[
      A^T = \begin{bmatrix} 2 & 1 & 3 \\ 1 & -2 & -1 \\ -1 & 0 & 1 \end{bmatrix}
    \]
    % A^T is the transpose of A, converting A from a 3x3 matrix to another 3x3 matrix.

    \[
      A^T A = \begin{bmatrix} 14 & -3 & 1 \\ -3 & 6 & -2 \\ 1 & -2 & 2 \end{bmatrix}
    \]
    % A^T A is the product of A^T and A, resulting in a 3x3 matrix.

    Then, calculate \( A^T b \):
    \[
      A^T b = \begin{bmatrix} 2 & 1 & 3 \\ 1 & -2 & -1 \\ -1 & 0 & 1 \end{bmatrix} \begin{bmatrix} 1 \\ 0 \\ 1 \end{bmatrix} = \begin{bmatrix} 5 \\ 0 \\ 0 \end{bmatrix}
    \]
    % A^T b is the product of A^T and b, resulting in a 3x1 matrix.

    Now, solve the normal equation:
    \[
      \begin{bmatrix} 14 & -3 & 1 \\ -3 & 6 & -2 \\ 1 & -2 & 2 \end{bmatrix} \begin{bmatrix} x_1 \\ x_2 \\ x_3 \end{bmatrix} = \begin{bmatrix} 5 \\ 0 \\ 0 \end{bmatrix}
    \]
    % Solving this system of linear equations to find x_1, x_2, and x_3.

    Thus, the least squares solution \( x \) is:
    \[
      x = \begin{bmatrix} 0.4 \\ 0.2 \\ 0.0 \end{bmatrix}
    \]
    % The solution provides the values for x_1, x_2, and x_3 that minimize the sum of the squares of the residuals in the equation Ax = b.
    \pagebreak

\end{enumerate}

\section*{Exercise 5.4.6}
Find the least squares solution to the linear systems in Exercise 5.4.1 under the weighted norm
\[
  \|x\|^2 = x_1^2 + 2x_2^2 + 3x_3^2
\]
% This norm gives different weights to the components of the vector x, influencing the least squares solution by emphasizing the minimization of the terms associated with higher weights.

We use the same systems as in 5.4.1 for each part (a), (b), and (c), and solve them considering the weighted norm.

\subsection*{Part (a)}
Given the same \( A \) and \( b \) as in 5.4.1(a):
\[
  A = \begin{bmatrix} 1 \\ 2 \\ 1 \end{bmatrix}, \quad b = \begin{bmatrix} 1 \\ 1 \\ 0 \end{bmatrix}
\]
% The solution process changes here because of the weighted norm.

First, we adjust the normal equation considering the weighted norm:
\[
  W = \begin{bmatrix} 1 & 0 & 0 \\ 0 & \sqrt{2} & 0 \\ 0 & 0 & \sqrt{3} \end{bmatrix}
\]
% W is the matrix representing the square roots of the weights in the norm.

\[
  \tilde{A} = WA, \quad \tilde{b} = Wb
\]
% We modify A and b by the weights to reflect their influence in the least squares solution.

Calculate \( \tilde{A}^T \tilde{A} \) and \( \tilde{A}^T \tilde{b} \):
\[
  \tilde{A} = \begin{bmatrix} 1 \\ 2\sqrt{2} \\ \sqrt{3} \end{bmatrix}, \quad \tilde{b} = \begin{bmatrix} 1 \\ \sqrt{2} \\ 0 \end{bmatrix}
\]
% Adjusting the vectors A and b according to the weights in the norm.

\[
  \tilde{A}^T \tilde{A} = \begin{bmatrix} 1 & 2\sqrt{2} & \sqrt{3} \end{bmatrix} \begin{bmatrix} 1 \\ 2\sqrt{2} \\ \sqrt{3} \end{bmatrix}, \quad \tilde{A}^T \tilde{b} = \begin{bmatrix} 1 & 2\sqrt{2} & \sqrt{3} \end{bmatrix} \begin{bmatrix} 1 \\ \sqrt{2} \\ 0 \end{bmatrix}
\]
% Calculating the products necessary for the weighted normal equation.

Now, solve the weighted normal equation for \( x \):
\[
  \tilde{A}^T \tilde{A} x = \tilde{A}^T \tilde{b}
\]
% Solving the adjusted normal equation which now takes the weights into account.

\pagebreak

\subsection*{Part (b)}
Given the same \( A \) and \( b \) as in 5.4.1(b):
\[
  A = \begin{bmatrix} 1 & 0 \\ 2 & -1 \\ 3 & 5 \end{bmatrix}, \quad b = \begin{bmatrix} 3 \\ 7 \\ 7 \end{bmatrix}
\]
% We need to find the weighted least squares solution, considering the specified norm.

As in part (a), first define the weight matrix \( W \) and adjust \( A \) and \( b \):
\[
  W = \begin{bmatrix} 1 & 0 & 0 \\ 0 & \sqrt{2} & 0 \\ 0 & 0 & \sqrt{3} \end{bmatrix}
\]
% W represents the square roots of the weights in the norm.

Adjust \( A \) and \( b \) to reflect the weights:
\[
  \tilde{A} = W A = \begin{bmatrix} 1 & 0 \\ 2\sqrt{2} & -\sqrt{2} \\ 3\sqrt{3} & 5\sqrt{3} \end{bmatrix}, \quad \tilde{b} = W b = \begin{bmatrix} 3 \\ 7\sqrt{2} \\ 7\sqrt{3} \end{bmatrix}
\]
% The adjustment factors in the weights for each row of A and b, reflecting the influence of the weighted norm.

Calculate \( \tilde{A}^T \tilde{A} \) and \( \tilde{A}^T \tilde{b} \):
\[
  \tilde{A}^T = \begin{bmatrix} 1 & 2\sqrt{2} & 3\sqrt{3} \\ 0 & -\sqrt{2} & 5\sqrt{3} \end{bmatrix}
\]
\[
  \tilde{A}^T \tilde{A} = \tilde{A}^T \tilde{A} = \begin{bmatrix} 1+8+27 & 0-2\sqrt{2}+15\sqrt{6} \\ 0-2\sqrt{2}+15\sqrt{6} & 2+75 \end{bmatrix}
\]
\[
  \tilde{A}^T \tilde{b} = \tilde{A}^T \tilde{b} = \begin{bmatrix} 1\cdot3 + 2\sqrt{2}\cdot7\sqrt{2} + 3\sqrt{3}\cdot7\sqrt{3} \\ 0\cdot3 - \sqrt{2}\cdot7\sqrt{2} + 5\sqrt{3}\cdot7\sqrt{3} \end{bmatrix}
\]
% We compute these matrix products to form the weighted normal equation.

Now solve the weighted normal equation:
\[
  \tilde{A}^T \tilde{A} x = \tilde{A}^T \tilde{b}
\]

\pagebreak

% Solving this equation will yield the vector x that minimizes the weighted sum of squares of the residuals.
\subsection*{Part (c)}
Given the same \( A \) and \( b \) as in 5.4.1(c):
\[
  A = \begin{bmatrix} 2 & 1 & -1 \\ 1 & -2 & 0 \\ 3 & -1 & 1 \end{bmatrix}, \quad b = \begin{bmatrix} 1 \\ 0 \\ 1 \end{bmatrix}
\]
% Again, the task is to find the weighted least squares solution with the specified norm.

As in previous parts, define the weight matrix \( W \) and adjust \( A \) and \( b \):
\[
  W = \begin{bmatrix} 1 & 0 & 0 \\ 0 & \sqrt{2} & 0 \\ 0 & 0 & \sqrt{3} \end{bmatrix}
\]
% W encodes the weights specified for the norm.

Adjust \( A \) and \( b \) accordingly:
\[
  \tilde{A} = W A = \begin{bmatrix} 2 & 1 & -1 \\ \sqrt{2} & -2\sqrt{2} & 0 \\ 3\sqrt{3} & -\sqrt{3} & \sqrt{3} \end{bmatrix}, \quad \tilde{b} = W b = \begin{bmatrix} 1 \\ 0 \\ \sqrt{3} \end{bmatrix}
\]
% Adjusting A and b reflects the influence of the weighted norm in the least squares solution.

Calculate \( \tilde{A}^T \tilde{A} \) and \( \tilde{A}^T \tilde{b} \):
\[
  \tilde{A}^T = \begin{bmatrix} 2 & \sqrt{2} & 3\sqrt{3} \\ 1 & -2\sqrt{2} & -\sqrt{3} \\ -1 & 0 & \sqrt{3} \end{bmatrix}
\]
\[
  \tilde{A}^T \tilde{A} = \begin{bmatrix} 4 + 2 + 27 & 2 - 2\sqrt{2} - 3\sqrt{3} & -2 + 0 + 3\sqrt{3} \\ 2 - 2\sqrt{2} - 3\sqrt{3} & 1 + 8 + 1 & -1 + 0 - 1 \\ -2 + 0 + 3\sqrt{3} & -1 + 0 - 1 & 1 + 0 + 3 \end{bmatrix}
\]
\[
  \tilde{A}^T \tilde{b} = \begin{bmatrix} 2\cdot1 + \sqrt{2}\cdot0 + 3\sqrt{3}\cdot\sqrt{3} \\ 1\cdot1 - 2\sqrt{2}\cdot0 - \sqrt{3}\cdot\sqrt{3} \\ -1\cdot1 + 0\cdot0 + \sqrt{3}\cdot\sqrt{3} \end{bmatrix}
\]
% Calculating these products to form the weighted normal equation.

Now solve the weighted normal equation:
\[
  \tilde{A}^T \tilde{A} x = \tilde{A}^T \tilde{b}
\]
% Solving this equation will find the vector x that minimizes the weighted sum of squares of the residuals.

\pagebreak

\section*{Exercise 5.5.4}
A 20-pound turkey that is at the room temperature of 72\(^{\circ}\)F is placed in the oven at 1:00 PM. The temperature of the turkey is observed in 20-minute intervals to be 79\(^{\circ}\)F, 88\(^{\circ}\)F, and 96\(^{\circ}\)F. A turkey is cooked when its temperature reaches 165\(^{\circ}\)F. How much longer do you need to wait until the turkey is done?

% We are provided with a sequence of temperatures at given times. We can use Newton's Law of Cooling to model the temperature change of the turkey.

\textbf{Solution:}
% First, we need to determine the rate of heating, which can be approximated using the exponential model of Newton's Law of Cooling: T(t) = T_{\text{env}} + (T_0 - T_{\text{env}}) e^{-kt}.

Given temperatures:
\[
  T_0 = 72, \quad T_{20} = 79, \quad T_{40} = 88, \quad T_{60} = 96 \quad \text{(all in degrees Fahrenheit)}
\]
% T_0 is the initial temperature, and T_20, T_40, and T_60 are the temperatures at 20, 40, and 60 minutes after placing the turkey in the oven.

% The environment temperature, T_{\text{env}}, is typically the oven temperature, which we'll assume remains constant and much higher than the turkey's temperature.

We need to find the constant \( k \) in the cooling equation. We can use the temperatures at different times to set up equations based on the model, and then solve for \( k \) using regression or a system of equations.

% Once k is determined, we predict the time \( t \) at which \( T(t) = 165 \).

To find \( k \), consider the equations:
\[
  \begin{aligned}
    79 & = T_{\text{env}} + (72 - T_{\text{env}}) e^{-20k}, \\
    88 & = T_{\text{env}} + (72 - T_{\text{env}}) e^{-40k}, \\
    96 & = T_{\text{env}} + (72 - T_{\text{env}}) e^{-60k}.
  \end{aligned}
\]
% These equations correspond to the temperature model at 20, 40, and 60 minutes.

Once \( k \) and \( T_{\text{env}} \) are determined, solve for \( t \) such that:
\[
  165 = T_{\text{env}} + (72 - T_{\text{env}}) e^{-kt}.
\]
% This equation will give us the time \( t \) when the turkey reaches the cooked temperature of 165 degrees Fahrenheit.

% Finally, the answer will tell us how much longer after the last observation time (60 minutes) it will take for the turkey to be done.
\pagebreak

\section*{Exercise 1.9.2}
Verify the determinant product formula when matrices \(A\) and \(B\) are given as:
\[
  A = \begin{bmatrix} 1 & -1 & 3 \\ 2 & -1 & 1 \\ 4 & -2 & 0 \end{bmatrix}, \quad B = \begin{bmatrix} 0 & 1 & -1 \\ 1 & -3 & -2 \\ 2 & 0 & 1 \end{bmatrix}
\]
% We are tasked with verifying that the determinant of the product of two matrices equals the product of their determinants, i.e., \det(AB) = \det(A) \det(B).

\textbf{Solution:}
% First, we calculate the determinants of A and B separately.

To find \( \det(A) \) and \( \det(B) \):
\[
  \det(A) = 1 \left((-1)(0) - (1)(-2)\right) - (-1) \left(2(0) - (1)(4)\right) + 3 \left(2(-2) - (-1)(4)\right)
\]
% This expression is derived using the cofactor expansion for determinant.

\[
  \det(B) = 0 \left((-3)(1) - (-2)(0)\right) + 1 \left((1)(1) - (-1)(2)\right) + (-1) \left(1(-2) - (-3)(2)\right)
\]
% Similarly, we use cofactor expansion for B.

% Next, we need to compute the determinant of the product AB, which involves matrix multiplication followed by determinant calculation.

First, compute the product \( AB \):
\[
  AB = A \times B = \begin{bmatrix}
    1\cdot0 + (-1)\cdot1 + 3\cdot2 & 1\cdot1 + (-1)(-3) + 3\cdot0 & 1\cdot(-1) + (-1)(-2) + 3\cdot1 \\
    2\cdot0 + (-1)\cdot1 + 1\cdot2 & 2\cdot1 + (-1)(-3) + 1\cdot0 & 2\cdot(-1) + (-1)(-2) + 1\cdot1 \\
    4\cdot0 + (-2)\cdot1 + 0\cdot2 & 4\cdot1 + (-2)(-3) + 0\cdot0 & 4\cdot(-1) + (-2)(-2) + 0\cdot1 \\
  \end{bmatrix}
\]
% The product AB is calculated by performing matrix multiplication between A and B.

Calculate \( \det(AB) \):
% We will use the cofactor expansion method similar to how we calculated \det(A) and \det(B).

\[
  \det(AB) = \text{[Cofactor expansion of the resultant matrix AB]}
\]
% This is left as an exercise for now.

Finally, verify if:
\[
  \det(AB) = \det(A) \times \det(B)
\]
% If the computed values match, the determinant product formula is verified.

\pagebreak

\section*{Exercise 8.2.1}
Find the eigenvalues and eigenvectors of the following matrices:

\subsection*{Part (a)}
Find the eigenvalues and eigenvectors for the matrix:
\[
  A = \begin{bmatrix} 1 & -2 \\ -2 & 1 \end{bmatrix}
\]

% Calculate the characteristic polynomial
\[
  \text{det}(A - \lambda I) = \text{det}\left(\begin{bmatrix} 1-\lambda & -2 \\ -2 & 1-\lambda \end{bmatrix}\right) = (1-\lambda)^2 - 4 = \lambda^2 - 2\lambda - 3
\]

% Solve for eigenvalues
\[
  \lambda^2 - 2\lambda - 3 = 0 \Rightarrow (\lambda - 3)(\lambda + 1) = 0 \Rightarrow \lambda_1 = 3, \lambda_2 = -1
\]

% Find eigenvectors for each eigenvalue
% For \lambda_1 = 3
\[
  (A - 3I)x = \begin{bmatrix} -2 & -2 \\ -2 & -2 \end{bmatrix}x = 0 \Rightarrow x_1 = t\begin{bmatrix} 1 \\ 1 \end{bmatrix}
\]

% For \lambda_2 = -1
\[
  (A + I)x = \begin{bmatrix} 2 & -2 \\ -2 & 2 \end{bmatrix}x = 0 \Rightarrow x_2 = s\begin{bmatrix} 1 \\ -1 \end{bmatrix}
\]

\pagebreak

\subsection*{Part (b)}
Find the eigenvalues and eigenvectors for the matrix:
\[
B = \begin{bmatrix} 1 & -\frac{2}{3} \\ \frac{1}{2} & -\frac{1}{6} \end{bmatrix}
\]

% Calculate the characteristic polynomial
\[
\text{det}(B - \lambda I) = \text{det}\left(\begin{bmatrix} 1-\lambda & -\frac{2}{3} \\ \frac{1}{2} & -\frac{1}{6}-\lambda \end{bmatrix}\right) = (1-\lambda)\left(-\frac{1}{6}-\lambda\right) + \frac{1}{3}
\]

% Expand and simplify to find the characteristic polynomial
\[
\lambda^2 - \frac{5}{6}\lambda - \frac{1}{6} = 0
\]

% Solve for eigenvalues using the quadratic formula
\[
\lambda = \frac{\frac{5}{6} \pm \sqrt{\left(\frac{5}{6}\right)^2 + \frac{4}{6}}}{2}
\]
\[
= \frac{\frac{5}{6} \pm \sqrt{\frac{49}{36}}}{2}
\]
\[
= \frac{\frac{5}{6} \pm \frac{7}{6}}{2}
\]
\[
\Rightarrow \lambda_1 = 1, \quad \lambda_2 = -\frac{1}{6}
\]

% Find eigenvectors for each eigenvalue
% For \lambda_1 = 1
\[
(B - I)x = \begin{bmatrix} 0 & -\frac{2}{3} \\ \frac{1}{2} & -\frac{7}{6} \end{bmatrix}x = 0 \Rightarrow x_1 = t\begin{bmatrix} 2 \\ 3 \end{bmatrix} \text{ (normalized)}
\]

% For \lambda_2 = -\frac{1}{6}
\[
(B + \frac{1}{6}I)x = \begin{bmatrix} \frac{7}{6} & -\frac{2}{3} \\ \frac{1}{2} & 0 \end{bmatrix}x = 0 \Rightarrow x_2 = s\begin{bmatrix} 2 \\ 3 \end{bmatrix} \text{ (normalized)}
\]

\pagebreak

\subsection*{Part (c)}
Find the eigenvalues and eigenvectors for the matrix:
\[
  C = \begin{bmatrix} 3 & 1 \\ -1 & 1 \end{bmatrix}
\]

% Calculate the characteristic polynomial
\[
  \text{det}(C - \lambda I) = \text{det}\left(\begin{bmatrix} 3-\lambda & 1 \\ -1 & 1-\lambda \end{bmatrix}\right) = (3-\lambda)(1-\lambda) + 1 = \lambda^2 - 4\lambda + 4
\]

% Solve for eigenvalues
\[
  \lambda^2 - 4\lambda + 4 = 0 \Rightarrow (\lambda - 2)^2 = 0 \Rightarrow \lambda = 2 \text{ (doubly repeated eigenvalue)}
\]

% Find eigenvectors for \lambda = 2
\[
  (C - 2I)x = \begin{bmatrix} 1 & 1 \\ -1 & -1 \end{bmatrix}x = 0 \Rightarrow x = t\begin{bmatrix} -1 \\ 1 \end{bmatrix}
\]

\pagebreak

\subsection*{Part (d)}
Find the eigenvalues and eigenvectors for the matrix:
\[
D = \begin{bmatrix} 1 & 2 \\ -1 & 1 \end{bmatrix}
\]

% Calculate the characteristic polynomial
\[
\text{det}(D - \lambda I) = \text{det}\left(\begin{bmatrix} 1-\lambda & 2 \\ -1 & 1-\lambda \end{bmatrix}\right) = (1-\lambda)^2 + 2
\]

% Expand and simplify to find the characteristic polynomial
\[
\lambda^2 - 2\lambda + 3 = 0
\]

% Solve for eigenvalues using the quadratic formula
\[
\lambda = \frac{2 \pm \sqrt{4 - 12}}{2}
\]
\[
= \frac{2 \pm \sqrt{-8}}{2}
\]
\[
= \frac{2 \pm 2i\sqrt{2}}{2}
\]
\[
\Rightarrow \lambda_1 = 1 + i\sqrt{2}, \quad \lambda_2 = 1 - i\sqrt{2}
\]

% Find eigenvectors for each eigenvalue
% For \lambda_1 = 1 + i\sqrt{2}
\[
(D - (1 + i\sqrt{2})I)x = \begin{bmatrix} -i\sqrt{2} & 2 \\ -1 & -i\sqrt{2} \end{bmatrix}x = 0 \Rightarrow x_1 = t\begin{bmatrix} i\sqrt{2} \\ 1 \end{bmatrix}
\]

% For \lambda_2 = 1 - i\sqrt{2}
\[
(D - (1 - i\sqrt{2})I)x = \begin{bmatrix} i\sqrt{2} & 2 \\ -1 & i\sqrt{2} \end{bmatrix}x = 0 \Rightarrow x_2 = s\begin{bmatrix} -i\sqrt{2} \\ 1 \end{bmatrix}
\]

\pagebreak

\subsection*{Part (e)}
Find the eigenvalues and eigenvectors for the matrix:
\[
  E = \begin{bmatrix} 3 & -1 & 0 \\ -1 & 2 & -1 \\ 0 & -1 & 3 \end{bmatrix}
\]

% Calculate the characteristic polynomial
\[
  \text{det}(E - \lambda I) = \text{det}\left(\begin{bmatrix} 3-\lambda & -1 & 0 \\ -1 & 2-\lambda & -1 \\ 0 & -1 & 3-\lambda \end{bmatrix}\right) = (3-\lambda)[(2-\lambda)(3-\lambda) - 1] + 1 = \lambda^3 - 8\lambda^2 + 21\lambda - 18
\]

% Solve for eigenvalues
\[
  \lambda^3 - 8\lambda^2 + 21\lambda - 18 = 0 \Rightarrow \lambda = 1, 2, 4
\]

% Find eigenvectors for each eigenvalue
% For \lambda = 1
\[
  (E - I)x = \begin{bmatrix} 2 & -1 & 0 \\ -1 & 1 & -1 \\ 0 & -1 & 2 \end{bmatrix}x = 0 \Rightarrow x_1 = t\begin{bmatrix} 1 \\ 2 \\ 1 \end{bmatrix}
\]

% For \lambda = 2
\[
  (E - 2I)x = \begin{bmatrix} 1 & -1 & 0 \\ -1 & 0 & -1 \\ 0 & -1 & 1 \end{bmatrix}x = 0 \Rightarrow x_2 = s\begin{bmatrix} 1 \\ 1 \\ 1 \end{bmatrix}
\]

% For \lambda = 4
\[
  (E - 4I)x = \begin{bmatrix} -1 & -1 & 0 \\ -1 & -2 & -1 \\ 0 & -1 & -1 \end{bmatrix}x = 0 \Rightarrow x_3 = u\begin{bmatrix} 1 \\ -2 \\ 1 \end{bmatrix}
\]

\pagebreak

\subsection*{Part (f)}
Find the eigenvalues and eigenvectors for the matrix:
\[
  F = \begin{bmatrix} -1 & -1 & 4 \\ 1 & 3 & -2 \\ 1 & 1 & -1 \end{bmatrix}
\]

% Calculate the characteristic polynomial
\[
  \text{det}(F - \lambda I) = \text{det}\left(\begin{bmatrix} -1-\lambda & -1 & 4 \\ 1 & 3-\lambda & -2 \\ 1 & 1 & -1-\lambda \end{bmatrix}\right) = -\lambda^3 + \lambda^2 + 5\lambda - 3
\]

% Solve for eigenvalues
\[
  -\lambda^3 + \lambda^2 + 5\lambda - 3 = 0 \Rightarrow \lambda = 3, -1, 1
\]

% Find eigenvectors for each eigenvalue
% For \lambda = 3
\[
  (F - 3I)x = \begin{bmatrix} -4 & -1 & 4 \\ 1 & 0 & -2 \\ 1 & 1 & -4 \end{bmatrix}x = 0 \Rightarrow x_1 = t\begin{bmatrix} 1 \\ 2 \\ 1 \end{bmatrix}
\]

% For \lambda = -1
\[
  (F + I)x = \begin{bmatrix} 0 & -1 & 4 \\ 1 & 4 & -2 \\ 1 & 1 & 0 \end{bmatrix}x = 0 \Rightarrow x_2 = s\begin{bmatrix} 4 \\ 0 \\ 1 \end{bmatrix}
\]

% For \lambda = 1
\[
  (F - I)x = \begin{bmatrix} -2 & -1 & 4 \\ 1 & 2 & -2 \\ 1 & 1 & -2 \end{bmatrix}x = 0 \Rightarrow x_3 = u\begin{bmatrix} 2 \\ 0 \\ 1 \end{bmatrix}
\]

\pagebreak

\subsection*{Part (g)}
Find the eigenvalues and eigenvectors for the matrix:
\[
  G = \begin{bmatrix} 1 & -3 & 11 \\ 2 & -6 & 16 \\ 1 & -3 & 7 \end{bmatrix}
\]

% Calculate the characteristic polynomial
\[
  \text{det}(G - \lambda I) = \text{det}\left(\begin{bmatrix} 1-\lambda & -3 & 11 \\ 2 & -6-\lambda & 16 \\ 1 & -3 & 7-\lambda \end{bmatrix}\right) = -\lambda^3 + 2\lambda^2 - \lambda + 2
\]

% Solve for eigenvalues
\[
  -\lambda^3 + 2\lambda^2 - \lambda + 2 = 0 \Rightarrow \lambda = -1, 1, 2
\]

% Find eigenvectors for each eigenvalue
% For \lambda = -1
\[
  (G + I)x = \begin{bmatrix} 2 & -3 & 11 \\ 2 & -5 & 16 \\ 1 & -3 & 8 \end{bmatrix}x = 0 \Rightarrow x_1 = t\begin{bmatrix} 1 \\ 2 \\ 0 \end{bmatrix}
\]

% For \lambda = 1
\[
  (G - I)x = \begin{bmatrix} 0 & -3 & 11 \\ 2 & -7 & 16 \\ 1 & -3 & 6 \end{bmatrix}x = 0 \Rightarrow x_2 = s\begin{bmatrix} 3 \\ -2 \\ 1 \end{bmatrix}
\]

% For \lambda = 2
\[
  (G - 2I)x = \begin{bmatrix} -1 & -3 & 11 \\ 2 & -8 & 16 \\ 1 & -3 & 5 \end{bmatrix}x = 0 \Rightarrow x_3 = u\begin{bmatrix} 1 \\ 0 \\ 0 \end{bmatrix}
\]

\pagebreak

\subsection*{Part (h)}
Find the eigenvalues and eigenvectors for the matrix:
\[
  H = \begin{bmatrix} 2 & -1 & -1 \\ -2 & 1 & 1 \\ 1 & 0 & 1 \end{bmatrix}
\]

% Calculate the characteristic polynomial
\[
  \text{det}(H - \lambda I) = \text{det}\left(\begin{bmatrix} 2-\lambda & -1 & -1 \\ -2 & 1-\lambda & 1 \\ 1 & 0 & 1-\lambda \end{bmatrix}\right) = -\lambda^3 + 4\lambda^2 - 5\lambda + 2
\]

% Solve for eigenvalues
\[
  -\lambda^3 + 4\lambda^2 - 5\lambda + 2 = 0 \Rightarrow \lambda = 1, 1, 2
\]

% Find eigenvectors for each eigenvalue
% For \lambda = 1 (repeated eigenvalue)
\[
  (H - I)x = \begin{bmatrix} 1 & -1 & -1 \\ -2 & 0 & 1 \\ 1 & 0 & 0 \end{bmatrix}x = 0 \Rightarrow x_1 = t\begin{bmatrix} 1 \\ -1 \\ 1 \end{bmatrix}
\]

% For \lambda = 2
\[
  (H - 2I)x = \begin{bmatrix} 0 & -1 & -1 \\ -2 & -1 & 1 \\ 1 & 0 & -1 \end{bmatrix}x = 0 \Rightarrow x_2 = s\begin{bmatrix} 1 \\ 1 \\ 1 \end{bmatrix}
\]

\pagebreak

\subsection*{Part (i)}
Find the eigenvalues and eigenvectors for the matrix:
\[
  I = \begin{bmatrix} -4 & -4 & 2 \\ 3 & 4 & -1 \\ -3 & -2 & 3 \end{bmatrix}
\]

% Calculate the characteristic polynomial
\[
  \text{det}(I - \lambda I) = \text{det}\left(\begin{bmatrix} -4-\lambda & -4 & 2 \\ 3 & 4-\lambda & -1 \\ -3 & -2 & 3-\lambda \end{bmatrix}\right) = -\lambda^3 + 3\lambda^2 + 11\lambda - 27
\]

% Solve for eigenvalues
\[
  -\lambda^3 + 3\lambda^2 + 11\lambda - 27 = 0 \Rightarrow \lambda = -3, 1, 9
\]

% Find eigenvectors for each eigenvalue
% For \lambda = -3
\[
  (I + 3I)x = \begin{bmatrix} -1 & -4 & 2 \\ 3 & 7 & -1 \\ -3 & -2 & 6 \end{bmatrix}x = 0 \Rightarrow x_1 = t\begin{bmatrix} 2 \\ 1 \\ 1 \end{bmatrix}
\]

% For \lambda = 1
\[
  (I - I)x = \begin{bmatrix} -5 & -4 & 2 \\ 3 & 3 & -1 \\ -3 & -2 & 2 \end{bmatrix}x = 0 \Rightarrow x_2 = s\begin{bmatrix} 4 \\ -2 \\ -3 \end{bmatrix}
\]

% For \lambda = 9
\[
  (I - 9I)x = \begin{bmatrix} -13 & -4 & 2 \\ 3 & -5 & -1 \\ -3 & -2 & -6 \end{bmatrix}x = 0 \Rightarrow x_3 = u\begin{bmatrix} 1 \\ -3 \\ 2 \end{bmatrix}
\]

\pagebreak

\subsection*{Part (j)}
Find the eigenvalues and eigenvectors for the matrix:
\[
J = \begin{bmatrix} 3 & 4 & 0 & 0 \\ 4 & 3 & 0 & 0 \\ 0 & 0 & 1 & 3 \\ 0 & 0 & 4 & 5 \end{bmatrix}
\]

% Calculate the characteristic polynomial
\[
\text{det}(J - \lambda I) = \text{det}\left(\begin{bmatrix} 3-\lambda & 4 & 0 & 0 \\ 4 & 3-\lambda & 0 & 0 \\ 0 & 0 & 1-\lambda & 3 \\ 0 & 0 & 4 & 5-\lambda \end{bmatrix}\right)
\]
\[
= ((3-\lambda)^2 - 16) \times ((1-\lambda)(5-\lambda) - 12)
\]

% Solve for eigenvalues
\[
= (\lambda^2 - 6\lambda + 25) \times (\lambda^2 - 6\lambda - 11)
\]
\[
= (\lambda - 7)(\lambda + 1)(\lambda - 1)(\lambda - 5)
\]
\[
\Rightarrow \lambda = 7, -1, 1, 5
\]

% Find eigenvectors for each eigenvalue
% For \lambda = 7
\[
(J - 7I)x = \begin{bmatrix} -4 & 4 & 0 & 0 \\ 4 & -4 & 0 & 0 \\ 0 & 0 & -6 & 3 \\ 0 & 0 & 4 & -2 \end{bmatrix}x = 0 \Rightarrow x_1 = t\begin{bmatrix} 1 \\ 1 \\ 0 \\ 0 \end{bmatrix}
\]

% For \lambda = -1
\[
(J + I)x = \begin{bmatrix} 4 & 4 & 0 & 0 \\ 4 & 4 & 0 & 0 \\ 0 & 0 & 2 & 3 \\ 0 & 0 & 4 & 6 \end{bmatrix}x = 0 \Rightarrow x_2 = s\begin{bmatrix} -1 \\ 1 \\ 0 \\ 0 \end{bmatrix}
\]

% For \lambda = 1
\[
(J - I)x = \begin{bmatrix} 2 & 4 & 0 & 0 \\ 4 & 2 & 0 & 0 \\ 0 & 0 & 0 & 3 \\ 0 & 0 & 4 & 4 \end{bmatrix}x = 0 \Rightarrow x_3 = u\begin{bmatrix} -2 \\ 2 \\ -1 \\ 1 \end{bmatrix}
\]

% For \lambda = 5
\[
(J - 5I)x = \begin{bmatrix} -2 & 4 & 0 & 0 \\ 4 & -2 & 0 & 0 \\ 0 & 0 & -4 & 3 \\ 0 & 0 & 4 & 0 \end{bmatrix}x = 0 \Rightarrow x_4 = v\begin{bmatrix} 1 \\ 1 \\ 1 \\ 1 \end{bmatrix}
\]

\pagebreak


\subsection*{Part (k)}
Find the eigenvalues and eigenvectors for the matrix:
\[
K = \begin{bmatrix} 4 & 0 & 0 & 0 \\ 1 & 3 & 0 & 0 \\ -1 & 1 & 2 & 0 \\ 1 & -1 & 1 & 1 \end{bmatrix}
\]

% Calculate the characteristic polynomial
\[
\text{det}(K - \lambda I) = \text{det}\left(\begin{bmatrix} 4-\lambda & 0 & 0 & 0 \\ 1 & 3-\lambda & 0 & 0 \\ -1 & 1 & 2-\lambda & 0 \\ 1 & -1 & 1 & 1-\lambda \end{bmatrix}\right)
\]
\[
= (4-\lambda) \times (3-\lambda) \times (2-\lambda) \times (1-\lambda)
\]

% Solve for eigenvalues
\[
\Rightarrow \lambda = 4, 3, 2, 1
\]

% Find eigenvectors for each eigenvalue
% For \lambda = 4
\[
(K - 4I)x = \begin{bmatrix} 0 & 0 & 0 & 0 \\ 1 & -1 & 0 & 0 \\ -1 & 1 & -2 & 0 \\ 1 & -1 & 1 & -3 \end{bmatrix}x = 0 \Rightarrow x_1 = t\begin{bmatrix} 1 \\ 0 \\ 0 \\ 0 \end{bmatrix}
\]

% For \lambda = 3
\[
(K - 3I)x = \begin{bmatrix} 1 & 0 & 0 & 0 \\ 1 & 0 & 0 & 0 \\ -1 & 1 & -1 & 0 \\ 1 & -1 & 1 & -2 \end{bmatrix}x = 0 \Rightarrow x_2 = s\begin{bmatrix} 0 \\ 1 \\ 0 \\ 0 \end{bmatrix}
\]

% For \lambda = 2
\[
(K - 2I)x = \begin{bmatrix} 2 & 0 & 0 & 0 \\ 1 & 1 & 0 & 0 \\ -1 & 1 & 0 & 0 \\ 1 & -1 & 1 & -1 \end{bmatrix}x = 0 \Rightarrow x_3 = u\begin{bmatrix} 0 \\ 0 \\ 1 \\ 0 \end{bmatrix}
\]

% For \lambda = 1
\[
(K - 1I)x = \begin{bmatrix} 3 & 0 & 0 & 0 \\ 1 & 2 & 0 & 0 \\ -1 & 1 & 1 & 0 \\ 1 & -1 & 1 & 0 \end{bmatrix}x = 0 \Rightarrow x_4 = v\begin{bmatrix} 0 \\ 0 \\ 0 \\ 1 \end{bmatrix}
\]


\end{document}
