\section*{If then statements}
\textbf{Format:} Proof. If A, then B.
\begin{enumerate}
    \item Assume A.
    \item Show that assuming A leads to B.
    \item Therefore, B is concluded from A.
\end{enumerate}
\textbf{Example:} 
\begin{proof}
Proof. If \(m = 1\), then \(m + 0 = 1\).
\begin{enumerate}
    \item Assume \(m = 1\).
    \item Considering \(m = 1\), we have \(1 + 0 = 1\).
    \item This simplifies to \(1 = 1\), which is true.
\end{enumerate}
\end{proof}

\section*{If then types}
Various types of implications and their representations:
\begin{itemize}
    \item \(A \Rightarrow B\): "If it is Wednesday, Dr. Beck will get a cup of coffee from the student union."
    \item \(B \Rightarrow A\) (Converse of the first): "If Dr. Beck got a cup of coffee from the student union, it is Wednesday."
    \item \(A \Leftrightarrow B\) (Bi-conditional, if and only if): "It is Wednesday if and only if Dr. Beck got a cup of coffee from the student union."
    \item \((\text{not } B) \Rightarrow (\text{not } A)\) (Contrapositive): "If Dr. Beck did not get a cup of coffee from the student union, then it is not Wednesday."
\end{itemize}

\section*{Induction Proof}
\textbf{Format:} 
\begin{proof}
Proof that \(F(x)\) is true for all \(x \in A\):
\begin{enumerate}
    \item \textbf{Base case:} Show \(F(a)\) is true, where \(a\) is the smallest element in \(A\).
    \item \textbf{Induction step:} Assume \(F(k)\) is true for an arbitrary \(k \in A\). Show that \(F(k) \Rightarrow F(k+1)\).
    \item Therefore, \(F(x)\) is true for all \(x \in A\).
\end{enumerate}
\end{proof}
\textbf{Example:} 
\begin{proof}
For all \(n \in \mathbb{N}\), \(n = n\).
\begin{enumerate}
    \item Base case (\(n = 1\)): \(1 = 1\) is true.
    \item Induction step: Assume \(n = n\) is true for an arbitrary natural number \(n\). Show that this implies \(n + 1 = n + 1\).
    \item By the induction hypothesis, \(n = n\). Adding 1 to both sides, \(n + 1 = n + 1\), which holds true.
\end{enumerate}
\end{proof}

\section*{Proof by contradiction}
\textbf{Format:}
\begin{proof}
Prove \(A\) is true by Contradiction:
\begin{enumerate}
    \item Assume \(A\) is false.
    \item Show that this assumption leads to a contradiction.
    \item Therefore, \(A\) must be true.
\end{enumerate}
\end{proof}
\textbf{Example:}
\begin{proof}
Prove there is no smallest negative integer.
\begin{enumerate}
    \item Assume, by way of contradiction, that there is a smallest negative integer, call it \(n\).
    \item Consider \(n - 1\). \(n - 1\) is also an integer and is smaller than \(n\), contradicting the assumption that \(n\) is the smallest negative integer.
    \item Therefore, there cannot be a smallest negative integer.
\end{enumerate}
\end{proof}
