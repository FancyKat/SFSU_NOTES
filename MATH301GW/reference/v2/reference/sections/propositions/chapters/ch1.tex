% Proposition 1.6
\section*{Proposition 1.6}
If $m$, $n$, and $p$ are integers, then $(m+n) \cdot p = mp+np$.

% Proposition 1.7
\section*{Proposition 1.7}
If $m$ is an integer, then $0+m = m$ and $1 \cdot m = m$.

% Proposition 1.8
\section*{Proposition 1.8}
If $m$ is an integer, then $(-m)+m = 0$.

% Proposition 1.9
\section*{Proposition 1.9}
Let $m$, $n$, and $p$ be integers. If $m+n = m+ p$, then $n = p$.

% Proposition 1.10
\section*{Proposition 1.10}
Let $m$, $x_1$, $x_2 \in \mathbb{Z}$. If $m$, $x_1$, $x_2$ satisfy the equation $m+x_1 = 0$ and $m+x_2 = 0$, then $x_1 = x_2$.

% Proposition 1.11
\section*{Proposition 1.11}
If $m$, $n$, $p$, and $q$ are integers, then:
\begin{enumerate}[label=(\roman*)]
    \item $(m+n)(p+q) = (mp+np)+(mq+nq)$.
    \item $m+(n+(p+q)) = (m+n)+(p+q) = ((m+n)+ p)+q$.
    \item $m+(n+ p) = (p+m)+n$.
    \item $m(np) = p(mn)$.
    \item $m(n+(p+q)) = (mn+mp)+mq$.
    \item $(m(n+ p))q = (mn)q+m(pq)$.
\end{enumerate}

% Proposition 1.12
\section*{Proposition 1.12}
Let $x \in \mathbb{Z}$. If $x$ has the property that for each integer $m$, $m+x = m$, then $x = 0$.

% Proposition 1.13
\section*{Proposition 1.13}
Let $x \in \mathbb{Z}$. If $x$ has the property that there exists an integer $m$ such that $m+x = m$, then $x = 0$.

% Proposition 1.14
\section*{Proposition 1.14}
For all $m \in \mathbb{Z}$, $m \cdot 0 = 0 = 0 \cdot m$.

% Proposition 1.16
\section*{Proposition 1.16}
If $m$ and $n$ are even integers, then so are $m+n$ and $mn$.

% Proposition 1.17
\section*{Proposition 1.17}
\begin{enumerate}[label=(\roman*)]
    \item $0$ is divisible by every integer.
    \item If $m$ is an integer not equal to $0$, then $m$ is not divisible by $0$.
\end{enumerate}

% Proposition 1.18
\section*{Proposition 1.18}
Let $x \in \mathbb{Z}$. If $x$ has the property that for all $m \in \mathbb{Z}$, $mx = m$, then $x=1$.

% Proposition 1.19
\section*{Proposition 1.19}
Let $x \in \mathbb{Z}$. If $x$ has the property that for some nonzero $m \in \mathbb{Z}$, $mx = m$, then $x = 1$.

% Proposition 1.20
\section*{Proposition 1.20}
For all $m, n \in \mathbb{Z}$, $(-m)(-n) = mn$.

% Corollary 1.21
\section*{Corollary 1.21}
$(-1)(-1) = 1$.

% Proposition 1.22
\section*{Proposition 1.22}
\begin{enumerate}[label=(\roman*)]
    \item For all $m \in \mathbb{Z}$, $-(−m) = m$.
    \item $-0 = 0$.
\end{enumerate}

% Proposition 1.23
\section*{Proposition 1.23}
Given $m, n \in \mathbb{Z}$, there exists one and only one $x \in \mathbb{Z}$ such that $m+x = n$.

% Proposition 1.24
\section*{Proposition 1.24}
Let $x \in \mathbb{Z}$. If $x \cdot x = x$ then $x = 0$ or $1$.

% Proposition 1.25
\section*{Proposition 1.25}
\begin{enumerate}[label=(\roman*)]
    \item $-(m+n) = (-m) + (-n)$.
    \item $-m = (-1)m$.
    \item $(-m)n = m(-n) = -(mn)$.
\end{enumerate}

% Proposition 1.26
\section*{Proposition 1.26}
Let $m, n \in \mathbb{Z}$. If $mn = 0$, then $m = 0$ or $n = 0$.

% Proposition 1.27
\section*{Proposition 1.27}
\begin{enumerate}[label=(\roman*)]
    \item $(m-n) + (p-q) = (m+p) - (n+q)$.
    \item $(m-n) - (p-q) = (m+q) - (n+p)$.
    \item $(m-n)(p-q) = (mp+nq) - (mq+np)$.
    \item $m-n = p-q$ if and only if $m+q = n+p$.
    \item $(m-n)p = mp - np$.
\end{enumerate}

