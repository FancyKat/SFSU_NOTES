\documentclass[12pt]{article}

% Packages
\usepackage[margin=1in]{geometry}
\usepackage{fancyhdr}
\usepackage{hyperref}
\usepackage{datetime} % For formatting the current date
\usepackage{parskip} % Adds space between paragraphs and removes paragraph indents
\usepackage{amsmath}
\usepackage{amssymb}

% Hyperlink settings
\hypersetup{
    colorlinks=true, % Colored links instead of boxes
    urlcolor=blue,   % Blue color for external links
}

% Custom date format
\newdateformat{monthyeardate}{%
\monthname[\THEMONTH] \THEYEAR}

% Header settings
\pagestyle{fancy}
\fancyhf{} % Clear all header and footer fields
\lhead{MATH 440 \\ 2:00 PM - 3:20 PM} % Left header with class number and time
\rhead{Marty Martin \\ \monthyeardate\today} % Right header with your name and the date
\renewcommand{\headrulewidth}{0.4pt} % Header underlining
\setlength{\headheight}{54pt} % Ensure there's enough space for two lines in the header

\begin{document}

\begin{center}
  \Large \textbf{Homework 12}
\end{center}

% Problem 3.12.4 - Moment-Generating Function for Discrete Random Variable
\section*{Problem 3.12.4}
\subsection*{Question}
% The question requires finding the moment-generating function for a discrete random variable with a geometric distribution.
Find the moment-generating function for the discrete random variable \(X\) whose probability function is given by:
\[
  p_X(k) = \left(\frac{3}{4}\right)^k \left(\frac{1}{4}\right), \quad k=0,1,2,\dots
\]

\subsection*{Solution}
% We compute the MGF by taking the sum of e^(t*k) times the probability function for each k.
\[
  M_X(t) = \sum_{k=0}^\infty e^{tk} \left(\frac{3}{4}\right)^k \left(\frac{1}{4}\right)
\]
% This sum is a geometric series, and we can use the formula for the sum of a geometric series to find the MGF.
\[
  M_X(t) = \frac{1}{4} \sum_{k=0}^\infty \left(e^t \frac{3}{4}\right)^k
\]
% The sum of a geometric series is 1 / (1 - r) where r is the common ratio.
\[
  M_X(t) = \frac{1}{4} \cdot \frac{1}{1 - \left(e^t \frac{3}{4}\right)} = \frac{1}{4 - 3e^t}
\]
\pagebreak

% Problem 3.12.8 - MGF for Continuous Random Variable
\section*{Problem 3.12.8}
\subsection*{Question}
% The question involves showing the MGF for a continuous random variable with an exponential distribution.
Let \(Y\) be a continuous random variable with \(f_Y(y) = ye^{-y}\) for \(0 \leq y \leq \infty\). Show that \(M_Y(t) = \frac{1}{(1-t)^2}\) for \(t < 1\).

\subsection*{Solution}
% Compute the MGF by integrating e^(t*y) times the PDF from 0 to infinity.
\[
  M_Y(t) = \int_0^\infty e^{ty} y e^{-y} \, dy
\]
% This integral can be simplified using integration by parts.
\[
  M_Y(t) = \int_0^\infty y e^{(t-1)y} \, dy
\]
% Applying integration by parts: let u = y and dv = e^{(t-1)y} dy.
\[
  u = y, \quad dv = e^{(t-1)y} dy
\]
\[
  du = dy, \quad v = \frac{e^{(t-1)y}}{t-1}
\]
\[
  M_Y(t) = y \frac{e^{(t-1)y}}{t-1} \bigg|_0^\infty - \int_0^\infty \frac{e^{(t-1)y}}{t-1} dy
\]
% Evaluate the boundary terms and the integral, ensuring t < 1 for convergence.
\[
  M_Y(t) = 0 - \left(-\frac{1}{(t-1)^2}\right) = \frac{1}{(1-t)^2}
\]
\pagebreak

% Problem 4.2.4 - Mutation Probability Calculation
\section*{Problem 4.2.4}
\subsection*{Question}
A chromosome mutation linked with colorblindness is known to occur, on average, once in every ten thousand births.
\begin{itemize}
    \item[(a)] Approximate the probability that exactly three of the next twenty thousand babies born will have the mutation.
    \item[(b)] How many babies out of the next twenty thousand would have to be born with the mutation to convince you that the "one in ten thousand" estimate is too low?
\end{itemize}

\subsection*{Solution}
% Detailed mathematical steps with factorial calculations and Poisson distribution applications.
% Part (a): Using the Poisson distribution to calculate the exact probability for three mutations.
\[
\text{(a)} \quad \lambda = \frac{20,000}{10,000} = 2  % The average number of mutations expected.
\]
\[
P(X = 3) = e^{-\lambda} \frac{\lambda^3}{3!} = e^{-2} \frac{2^3}{3!} = e^{-2} \frac{8}{6} = \frac{4}{3} e^{-2}
\]
% This expression uses the Poisson formula where lambda is the mean number of events (mutations).

% Part (b): Calculating a threshold to indicate an underestimate.
% We use the cumulative Poisson distribution to find when the observed number significantly exceeds expectation.
\[
\text{(b)} \quad \text{Let } k \text{ be the number where } P(X > k) \text{ indicates an underestimation.}
\]
% Calculation for P(X > 5) using the sum of probabilities for X = 0 to 5.
\[
P(X > 5) = 1 - (P(X = 0) + P(X = 1) + P(X = 2) + P(X = 3) + P(X = 4) + P(X = 5))
\]
\[
P(X = 0) = e^{-2} \frac{2^0}{0!}, \quad P(X = 1) = e^{-2} \frac{2^1}{1!}, \quad P(X = 2) = e^{-2} \frac{2^2}{2!}
\]
\[
P(X = 3) = \frac{4}{3} e^{-2}, \quad P(X = 4) = e^{-2} \frac{2^4}{4!}, \quad P(X = 5) = e^{-2} \frac{2^5}{5!}
\]
\[
P(X > 5) = 1 - \left( e^{-2} (1 + 2 + 2 + \frac{4}{3} + \frac{8}{3} + \frac{16}{15}) \right)
\]
\pagebreak


% Problem 4.2.6 - Probability of Large Insurance Claims
\section*{Problem 4.2.6}
\subsection*{Question}
A newly formed life insurance company has underwritten term policies on 120 women between the ages of forty and forty-four. Suppose that each woman has a 1/150 probability of dying during the next calendar year, and that each death requires the company to pay out \$50,000 in benefits. Approximate the probability that the company will have to pay at least \$150,000 in benefits next year.

\subsection*{Solution}
% Explanation of using the Poisson approximation
\[
\lambda = \frac{120}{150} = 0.8 \quad \text{(expected number of deaths)}
\]
% Calculate the minimum number of deaths to exceed $150,000 in benefits
\[
\text{Minimum number of deaths} = \frac{150,000}{50,000} = 3
\]
% Poisson probability calculation for at least 3 deaths
\[
P(X \geq 3) = 1 - P(X < 3) = 1 - (P(X=0) + P(X=1) + P(X=2))
\]
\[
P(X = 0) = e^{-0.8} \frac{0.8^0}{0!}, \quad P(X = 1) = e^{-0.8} \frac{0.8^1}{1!}, \quad P(X = 2) = e^{-0.8} \frac{0.8^2}{2!}
\]
\[
P(X \geq 3) = 1 - (e^{-0.8} + 0.8 e^{-0.8} + 0.32 e^{-0.8}) = 1 - 2.12 e^{-0.8}
\]


\end{document}
