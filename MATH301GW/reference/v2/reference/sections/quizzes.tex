\section*{Set Definition and Inclusion}

\textbf{(a) Let \( A \) and \( B \) be sets. Carefully define \( A \subseteq B \).}

\textit{A set \( A \) is a subset of a set \( B \), denoted \( A \subseteq B \),} means that if an element \( x \) is in \( A \), then that element \( x \) must also be in \( B \).

\textbf{(b) Carefully define \( A = B \).}

\textit{Two sets \( A \) and \( B \) are equal, denoted \( A = B \),} means that every element \( x \) in \( A \) is also in \( B \) and vice versa.

\section*{Definition of Division (m|n)}

\textbf{(a) Let \( m, n \in \mathbb{Z} \). Carefully define what it means that \( m \) divides \( n \).}

We say that \( m \) divides \( n \), denoted as \( m | n \), if there exists an integer \( j \) such that \( n = jm \).

\textbf{(b) Carefully define what it means for \( n \) to be even.}

An integer \( n \) is even if there exists an integer \( j \) such that \( n = 2j \), where \( 2 \) is defined as the sum \( 1 + 1 \) and \( 1 \) is established as the multiplicative identity.

\section*{Empty Set Definition and Subset Property}

\textbf{(a) Carefully define the empty set \( \emptyset \).}

\textit{The empty set \( \emptyset \)} is the unique set that contains no elements.

\textbf{(b) Explain why \( \emptyset \subseteq S \) for any set \( S \).}

The statement \textit{\( \emptyset \subseteq S \)} holds true for any set \( S \) because the condition "if \( x \) is in \( \emptyset \), then \( x \) is in \( S \)" is vacuously true due to the absence of any elements in \( \emptyset \).

\section*{Union and Intersection Definitions}

\textbf{(a) Let \( A \) and \( B \) be sets. Carefully define \( A \cup B \).}

The union \( A \cup B \) is defined as the set of elements that are in either \( A \), \( B \), or in both.

\textbf{(b) Carefully define \( A \cap B \).}

The intersection \( A \cap B \) is the set of elements that are in both \( A \) and \( B \).

\section*{Equivalence Relation Definition}

\textbf{Let \( \sim \) be a relation on a set \( A \). Carefully define what it means for \( \sim \) to be an equivalence relation.}

An equivalence relation \( \sim \) on a set \( A \) satisfies three conditions:
\begin{enumerate}
    \item Reflexivity: \textit{For all \( a \in A \), \( a \sim a \)}.
    \item Symmetry: \textit{For all \( a, b \in A \), if \( a \sim b \), then \( b \sim a \)}.
    \item Transitivity: \textit{For all \( a, b, c \in A \), if \( a \sim b \) and \( b \sim c \), then \( a \sim c \)}.
\end{enumerate}

\section*{Birthday Statement and Subtraction Definition}

\textbf{(a) Decide whether or not the statement "today is Wednesday or it's my birthday" is true.}

This statement is true because in mathematics, the logical "or" is inclusive. If either condition is met, the entire statement holds true.

\textbf{(b) Let \( m, n \in \mathbb{Z} \). Carefully define \( m - n \).}

The subtraction of \( n \) from \( m \), denoted \( m - n \), is defined as the addition of \( m \) to the additive inverse of \( n \): \( m + (-n) \).

\section*{Further Explanation on the Empty Set}

\textbf{(a) Carefully define the empty set \( \emptyset \):}

\textit{The empty set \( \emptyset \)} is a set that contains no elements whatsoever.

\textbf{(b) Explain why \( \emptyset \subseteq S \) for any set \( S \).}

The statement \( \emptyset \subseteq S \) is true for any set \( S \) because the premise "if \( x \) is in \( \emptyset \)" is never true, and therefore, the conditional statement "if \( x \) is in \( \emptyset \), then \( x \) is in \( S \)" is vacuously true.

\section*{Union and Intersection Definitions}

\textbf{(a) Let \( A \) and \( B \) be sets. Carefully define \( A \cup B \).}

The union of two sets \( A \) and \( B \), denoted by \( A \cup B \), is the set that includes all the elements that are either in \( A \), in \( B \), or in both.

\textbf{(b) Carefully define \( A \cap B \).}

The intersection of two sets \( A \) and \( B \), denoted by \( A \cap B \), is the set consisting of all elements that are both in \( A \) and \( B \).

\section*{Equivalence Relation Definition}

\textbf{Let \( \sim \) be a relation on a set \( A \). Carefully define what it means for \( \sim \) to be an equivalence relation.}

An equivalence relation on a set \( A \), denoted by \( \sim \), must satisfy the following conditions:
\begin{enumerate}
    \item \textbf{Reflexivity:} Every element is related to itself; that is, \( a \sim a \) for all \( a \in A \).
    \item \textbf{Symmetry:} If one element is related to another, then the second element is related to the first; in other words, if \( a \sim b \), then \( b \sim a \) for all \( a, b \in A \).
    \item \textbf{Transitivity:} If an element is related to a second element, which is in turn related to a third, then the first element is related to the third; that is, if \( a \sim b \) and \( b \sim c \), then \( a \sim c \) for all \( a, b, c \in A \).
\end{enumerate}

\section*{Logical Statements and Subtraction Definition}

\textbf{(a) Decide whether or not the statement "today is Wednesday or it's my birthday" is true.}

This statement can be considered true if today is indeed Wednesday, as the 'or' in the statement is inclusive. Therefore, even if it is not the speaker's birthday, the statement is still true if today is Wednesday.

\textbf{(b) Let \( m, n \in \mathbb{Z} \). Carefully define \( m - n \).}

Subtraction in the context of integers is defined by the operation \( m - n = m + (-n) \), where \( -n \) represents the additive inverse of \( n \), such that \( n + (-n) = 0 \).

\section*{Further Discussion on the Empty Set}

\textbf{(a) Carefully define the empty set \( \emptyset \):}

\textit{The empty set \( \emptyset \)} is the set with no elements. It is the unique set for which the statement "there exists an \( x \) such that \( x \) is in \( \emptyset \)" is always false.

\textbf{(b) Explain why \( \emptyset \subseteq S \) for any set \( S \).}

For any set \( S \), the empty set \( \emptyset \) is a subset because there are no elements in \( \emptyset \) to contradict the statement "if \( x \) is in \( \emptyset \), then \( x \) is in \( S \)." Hence, the statement \( \emptyset \subseteq S \) is always true.
