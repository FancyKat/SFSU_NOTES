\documentclass[12pt]{article}

% Packages
\usepackage[margin=1in]{geometry}
\usepackage{fancyhdr}
\usepackage{hyperref}
\usepackage{datetime} % For formatting the current date
\usepackage{parskip} % Adds space between paragraphs and removes paragraph indents
\usepackage{amsmath}
\usepackage{amssymb}

% Hyperlink settings
\hypersetup{
    colorlinks=true, % Colored links instead of boxes
    urlcolor=blue,   % Blue color for external links
}

% Custom date format
\newdateformat{monthyeardate}{%
\monthname[\THEMONTH] \THEYEAR}

% Header settings
\pagestyle{fancy}
\fancyhf{} % Clear all header and footer fields
\lhead{MATH 440 \\ 2:00 PM - 3:20 PM} % Left header with class number and time
\rhead{Marty Martin \\ \monthyeardate\today} % Right header with your name and the date
\renewcommand{\headrulewidth}{0.4pt} % Header underlining
\setlength{\headheight}{54pt} % Ensure there's enough space for two lines in the header

\begin{document}

\begin{center}
  \Large \textbf{Homework 8}
\end{center}

% Problem 3.7.12 - Concise Calculation of PDF for Points Inside a Circle
\section*{Problem 3.7.12}
\subsection*{Question}
A point is chosen at random from the interior of a circle whose equation is \(x^2 + y^2 \leq 4\). Let the random variables \(X\) and \(Y\) denote the \(x\)- and \(y\)-coordinates of the sampled point. Find the joint pdf \(f_{X,Y}(x, y)\).

\subsection*{Solution}
% Calculation of the area of the circle
The area \(A\) of the circle is calculated as:
\[
A = \pi r^2 = \pi \times 2^2 = 4\pi
\]

% Definition of the joint PDF
The joint pdf \(f_{X,Y}(x, y)\) is given by:
\[
f_{X,Y}(x, y) = \begin{cases} 
\frac{1}{4\pi} & \text{if } x^2 + y^2 \leq 4, \\
0 & \text{otherwise}.
\end{cases}
\]

% Verification of the normalization of the PDF
The integral of the pdf over the circle should equal 1:
\[
\int \int_{x^2 + y^2 \leq 4} \frac{1}{4\pi} \, dx \, dy = \frac{1}{4\pi} \times 4\pi = 1
\]

% Problem 3.7.14 - Probability Calculation with Continuous PDF
\section*{Problem 3.7.14}
\subsection*{Question}
Suppose that five independent observations are drawn from the continuous pdf \(f_T(t) = 2t\), \(0 \leq t \leq 1\). Let \(X\) denote the number of \(t\)'s that fall in the interval \(0 \leq t \leq \frac{1}{3}\) and let \(Y\) denote the number of \(t\)'s that fall in the interval \(\frac{1}{3} \leq t \leq \frac{2}{3}\). Find \(P_{XY}(1, 2)\).

\subsection*{Solution}
% Calculating probabilities for given intervals using the given pdf
\[
P(0 \leq T \leq \frac{1}{3}) = \int_0^{\frac{1}{3}} 2t \, dt = \left[ t^2 \right]_0^{\frac{1}{3}} = \left(\frac{1}{3}\right)^2 = \frac{1}{9},
\]
\[
P(\frac{1}{3} \leq T \leq \frac{2}{3}) = \int_{\frac{1}{3}}^{\frac{2}{3}} 2t \, dt = \left[ t^2 \right]_{\frac{1}{3}}^{\frac{2}{3}} = \left(\frac{4}{9}\right) - \left(\frac{1}{9}\right) = \frac{3}{9} = \frac{1}{3}.
\]
% Using the binomial distribution since observations are independent
\[
P_{XY}(1, 2) = \binom{5}{1} \left(\frac{1}{9}\right) \binom{4}{2} \left(\frac{1}{3}\right)^2 \left(\frac{5}{9}\right)^2,
\]
\[
= 5 \cdot \frac{1}{9} \cdot 6 \cdot \frac{1}{9} \cdot \left(\frac{25}{81}\right) = \frac{750}{6561}.
\]


\end{document}
