\documentclass[12pt]{article}

% Packages
\usepackage[margin=1in]{geometry}
\usepackage{fancyhdr}
\usepackage{hyperref}
\usepackage{datetime} % For formatting the current date
\usepackage{parskip} % Adds space between paragraphs and removes paragraph indents
\usepackage{amsmath}
\usepackage{amssymb}

% Hyperlink settings
\hypersetup{
    colorlinks=true, % Colored links instead of boxes
    urlcolor=blue,   % Blue color for external links
}

% Custom date format
\newdateformat{monthyeardate}{%
\monthname[\THEMONTH] \THEYEAR}

% Header settings
\pagestyle{fancy}
\fancyhf{} % Clear all header and footer fields
\lhead{MATH 440 \\ 2:00 PM - 3:20 PM} % Left header with class number and time
\rhead{Marty Martin \\ \monthyeardate\today} % Right header with your name and the date
\renewcommand{\headrulewidth}{0.4pt} % Header underlining
\setlength{\headheight}{54pt} % Ensure there's enough space for two lines in the header

\begin{document}

\begin{center}
  \Large \textbf{Homework 9}
\end{center}

% Problem 3.8.2 - Transformation of Random Variables
\section*{Problem 3.8.2}
\subsection*{Question}
Let \(f_Y(y) = \frac{3(1+y^2)}{14}\), \(0 \leq y \leq 2\). Define the random variable \(W\) by \(W = 3Y + 2\). Find \(f_W(w)\). Be sure to specify the values of \(w\) for which \(f_W(w) \neq 0\).

\subsection*{Solution}
% Applying transformation of variables technique to find f_W(w).
\[
  Y = \frac{W - 2}{3}, \quad f_W(w) = f_Y\left(\frac{w-2}{3}\right) \left| \frac{d}{dw}\left(\frac{w-2}{3}\right) \right|
\]
\[
  f_W(w) = \frac{3\left(1 + \left(\frac{w-2}{3}\right)^2\right)}{14} \cdot \frac{1}{3}
\]
% Simplifying the expression.
\[
  f_W(w) = \frac{1 + \left(\frac{w-2}{3}\right)^2}{14}
\]
% Valid for the transformed range of Y.
\[
  w: 2 \leq w \leq 8.
\]
\pagebreak


% Problem 3.8.6 - Independence of V from X + Y
\section*{Problem 3.8.6}
\subsection*{Question}
If a random variable \(V\) is independent of two dependent random variables \(X\) and \(Y\), prove that \(V\) is independent of \(X + Y\).

\subsection*{Proof}
% Simplified proof with only necessary mathematical steps and equations aligned for clarity.
Given \( V \) is independent of \( X \) and \( Y \),
\[
P(V \leq v, X + Y \leq x) = P(V \leq v) P(X + Y \leq x).
\]
By the definition of independence and the law of total probability,
\begin{align*}
P(V \leq v, X + Y \leq x) &= \int P(V \leq v, X + Y \leq x \mid X = t) f_X(t) \, dt \\
                          &= \int P(V \leq v) P(X + Y \leq x \mid X = t) f_X(t) \, dt \\
                          &= P(V \leq v) \int P(X + Y \leq x \mid X = t) f_X(t) \, dt.
\end{align*}
Since \( P(X + Y \leq x \mid X = t) \) simplifies to \( P(Y \leq x - t) \) by the definition of conditional probability,
\begin{align*}
\int P(X + Y \leq x \mid X = t) f_X(t) \, dt &= \int P(Y \leq x - t) f_X(t) \, dt \\
                                              &= P(X + Y \leq x).
\end{align*}
Thus,
\[
P(V \leq v, X + Y \leq x) = P(V \leq v) P(X + Y \leq x).
\]
This confirms \( V \) is independent of \( X + Y \).
\pagebreak


% Problem 3.8.8 - PDF of Transformed Uniform Variable
\section*{Problem 3.8.8}
\subsection*{Question}
Let \(Y\) be a uniform random variable over the interval \([0,1]\). Find the pdf of \(W = Y^2\).

\subsection*{Solution}
% Since Y is uniformly distributed, transforming Y to W = Y^2 requires the change of variables formula.
\[
  f_Y(y) = 1 \text{ for } 0 \leq y \leq 1, \quad W = Y^2 \Rightarrow Y = \sqrt{W}
\]
% Calculating the derivative of the transformation.
\[
  f_W(w) = f_Y(\sqrt{w}) \left| \frac{d}{dw}(\sqrt{w}) \right| = 1 \cdot \frac{1}{2\sqrt{w}}
\]
% The final form of the pdf.
\[
  f_W(w) = \frac{1}{2\sqrt{w}} \text{ for } 0 \leq w \leq 1.
\]
\pagebreak

% Problem 3.9.4 - Expected Score in a Marksmanship Competition
\section*{Problem 3.9.4}
\subsection*{Question}
Marksmanship competition at a certain level requires each contestant to take ten shots with each of two different handguns. Final scores are computed by taking a weighted average of four times the number of bull's-eyes made with the first gun plus six times the number gotten with the second. If Cathie has a 30\% chance of hitting the bull's-eye with each shot from the first gun and a 40\% chance with each shot from the second gun, what is her expected score?

\subsection*{Solution}
% Introducing the problem and the variables
We denote the number of bull's-eyes made with the first gun as \(X_1\) and with the second gun as \(X_2\). The variable \(X_1\) follows a binomial distribution with parameters \(n = 10\) (shots) and \(p = 0.3\) (probability of success), and similarly for \(X_2\) with \(p = 0.4\).

% Calculation of the expected number of bull's-eyes for each gun.
\[
  E(X_1) = n \times p_1 = 10 \times 0.3 = 3, \quad E(X_2) = 10 \times 0.4 = 4
\]
% These expectations come from the definition of the expected value for a binomial distribution.

% Calculating the final expected score based on the weights given.
The competition score is calculated using the formula:
\[
  \text{Score} = 4 \times X_1 + 6 \times X_2
\]
% Taking the expectation of the score.
\[
  E(\text{Score}) = E(4 \times X_1 + 6 \times X_2)
\]
% By the linearity of expectation, this becomes:
\[
  E(\text{Score}) = 4 \times E(X_1) + 6 \times E(X_2) = 4 \times 3 + 6 \times 4
\]
\[
  = 12 + 24 = 36
\]
% Conclusion of the expected score calculation.
Thus, Cathie's expected score in the competition is 36 points.

\pagebreak

% Problem 3.9.10 - Expected Value of Distance Squared
\section*{Problem 3.9.10}
\subsection*{Question}
Suppose that \(X\) and \(Y\) are both uniformly distributed over the interval \([0, 1]\). Calculate the expected value of the square of the distance of the random point \((X, Y)\) from the origin; that is, find \(E(X^2 + Y^2)\).

\subsection*{Solution}
% We use the properties of expectation and independence of X and Y.
\[
  E(X^2 + Y^2) = E(X^2) + E(Y^2)
\]
% For uniformly distributed variables on [0, 1], the expected value of the square is calculated.
\[
  E(X^2) = \int_0^1 x^2 \, dx = \frac{1}{3}, \quad E(Y^2) = \int_0^1 y^2 \, dy = \frac{1}{3}
\]
% The expected value of the sum of the squares.
\[
  E(X^2 + Y^2) = \frac{1}{3} + \frac{1}{3} = \frac{2}{3}
\]
\pagebreak

\end{document}
