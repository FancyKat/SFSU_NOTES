% Proposition 2.2
\section*{Proposition 2.2:}
For every $m \in \mathbb{Z}$, one and only one of the following is true: $m \in \mathbb{N}$, $-m \in \mathbb{N}$, or $m = 0$.
\begin{proof}
    \textit{Consider an arbitrary $m \in \mathbb{Z}$. We'll analyze each possibility separately.}
    \begin{align*}
        & \text{If $m > 0$, then by definition, $m$ is a natural number.} & & \text{(Definition of $\mathbb{N}$)} \\
        & \text{If $m < 0$, consider $-m$. Since $-m > 0$, by definition, $-m$ is a natural number.} & & \text{(Definition of $\mathbb{N}$)} \\
        & \text{If $m = 0$, it is neither positive nor negative, making it distinct from natural numbers.} & & \text{(Axiom 1.3)} \\
    \end{align*}
    \textit{These scenarios are mutually exclusive, confirming the proposition's statement.}
\end{proof}



% Proposition 2.3
\section*{Proposition 2.3:}
$1 \in \mathbb{N}$.
\begin{proof}
    \textit{The assertion $1 \in \mathbb{N}$ is foundational to the definition of natural numbers:}
    \begin{align*}
        & \text{$1$ is recognized as the first natural number.} & & \text{(Definition of $\mathbb{N}$)}
    \end{align*}
    \textit{Thus, by definition, $1$ is an element of $\mathbb{N}$.}
\end{proof}


% Proposition 2.4
\section*{Proposition 2.4:}
Let $m,n,p \in \mathbb{Z}$. If $m < n$ and $n < p$, then $m < p$.
\begin{proof}
    \textit{Given $m, n, p \in \mathbb{Z}$ with $m < n$ and $n < p$, we apply the transitive property:}
    \begin{align*}
        & \text{The transitivity of inequalities implies that if $m < n$ and $n < p$, then $m < p$.} & & \text{(Axiom 1.1)}
    \end{align*}
    \textit{Therefore, we conclude $m < p$.}
\end{proof}


% Proposition 2.5
\section*{Proposition 2.5:}
For each $n \in \mathbb{N}$, there exists $m \in \mathbb{N}$ such that $m > n$.
\begin{proof}
    \textit{Select any $n \in \mathbb{N}$. To find $m \in \mathbb{N}$ where $m > n$, we choose $n + 1$:}
    \begin{align*}
        & \text{Consider $m = n + 1$. The natural numbers are closed under addition, so $m \in \mathbb{N}$.} & & \text{(Axiom 2.1(i))} \\
        & \text{Clearly, $m = n + 1 > n$.} & & \\
    \end{align*}
    \textit{This confirms the existence of such an $m$ for every $n \in \mathbb{N}$.}
\end{proof}


% Proposition 2.6
\section*{Proposition 2.6:}
Let $m,n \in \mathbb{Z}$. If $m \leq n \leq m$, then $m = n$.
\begin{proof}
    \textit{Assuming $m \leq n$ and $n \leq m$ for integers $m$ and $n$:}
    \begin{align*}
        & \text{The relationship $m \leq n \leq m$ directly leads to $m = n$ by the properties of order in $\mathbb{Z}$.} & & \text{(Definition of order)} \\
    \end{align*}
    \textit{Thus, we establish that $m = n$.}
\end{proof}


% Proposition 2.7
\section*{Proposition 2.7:}
\begin{enumerate}[label=(\roman*)]
    \item If $m < n$, then $m+p < n+p$.
    \item If $m < n$ and $p < q$, then $m+p < n+q$.
    \item If $0 < m < n$ and $0 < p \leq q$, then $mp < nq$.
    \item If $m < n$ and $p < 0$, then $np < mp$.
\end{enumerate}
\section*{Proposition 2.7 (i)}
\textit{If $m < n$, then $m+p < n+p$ for all $m, n, p \in \mathbb{Z}$.}
\begin{proof}
    \textit{Given $m, n, p \in \mathbb{Z}$ with $m < n$:}
    \begin{align*}
        & \text{Adding $p$ to both $m$ and $n$ preserves the inequality:} & & \\
        & \text{$m + p < n + p$, by the properties of integer addition.} & & \text{(Axiom 1.1(ii))} \\
    \end{align*}
    \textit{Thus, we establish that $m + p < n + p$.}
\end{proof}

\section*{Proposition 2.7 (ii)}
\textit{If $m < n$ and $p < q$, then $m + p < n + q$ for all $m, n, p, q \in \mathbb{Z}$.}
\begin{proof}
    \textit{Given $m < n$ and $p < q$:}
    \begin{align*}
        & \text{Adding $m$ to $p$ and $n$ to $q$ and using the properties of inequalities:} & & \\
        & \text{$m + p < n + p$ and $n + p < n + q$, thus $m + p < n + q$.} & & \text{(Axiom 1.1(ii))} \\
    \end{align*}
    \textit{Therefore, $m + p < n + q$ is proven.}
\end{proof}

\section*{Proposition 2.7 (iii)}
\textit{If $0 < m < n$ and $0 < p \leq q$, then $mp < nq$ for all $m, n, p, q \in \mathbb{Z}$.}
\begin{proof}
    \textit{Given $0 < m < n$ and $0 < p \leq q$:}
    \begin{align*}
        & \text{Multiplying $m < n$ by $p$ and noting $p < q$, we get $mp < np \leq nq$.} & & \text{(Axiom 2.1(ii))} \\
    \end{align*}
    \textit{Hence, it follows that $mp < nq$.}
\end{proof}

\section*{Proposition 2.7 (iv)}
\textit{If $m < n$ and $p < 0$, then $np < mp$ for all $m, n, p \in \mathbb{Z}$.}
\begin{proof}
    \textit{Assume $m < n$ and $p < 0$:}
    \begin{align*}
        & \text{Multiplying the inequality $m < n$ by the negative number $p$ reverses the inequality:} & & \\
        & \text{Thus, $mp > np$, as multiplying by a negative number inverts the inequality.} & & \text{(Axiom 1.1(ii))} \\
    \end{align*}
    \textit{Therefore, we confirm that $np < mp$.}
\end{proof}



% Proposition 2.8
\section*{Proposition 2.8:}
Let $m,n \in \mathbb{Z}$. Exactly one of the following is true: $m < n$, $m = n$, $m > n$.
\begin{proof}
    \textit{For any $m, n \in \mathbb{Z}$, the trichotomy law ensures one and only one relation holds:}
    \begin{align*}
        & \text{The integers are well-ordered, ensuring that $m < n$, $m = n$, or $m > n$.} & & \text{(Axiom 1.1)} \\
    \end{align*}
    \textit{This exclusivity confirms the statement of the proposition.}
\end{proof}



% Proposition 2.9
\section*{Proposition 2.9:}
Let $m \in \mathbb{Z}$. If $m \neq 0$ then $m^2 \in \mathbb{N}$.
\begin{proof}
    \textit{Consider a non-zero $m \in \mathbb{Z}$:}
    \begin{align*}
        & \text{If $m > 0$, then $m^2 > 0$ and is natural. If $m < 0$, then $m^2 > 0$ as well.} & & \text{(Definition of $\mathbb{N}$)} \\
    \end{align*}
    \textit{Thus, in either case, $m^2$ is positive and belongs to $\mathbb{N}$.}
\end{proof}



% Proposition 2.10
\section*{Proposition 2.10:}
The equation $x^2 = -1$ has no solution in $\mathbb{Z}$.
\begin{proof}
    \textit{Assume for contradiction there exists an $x \in \mathbb{Z}$ where $x^2 = -1$:}
    \begin{align*}
        & \text{This implies $x^2$ is negative, contradicting the property that squares are non-negative.} & & \text{(Contradiction)} \\
    \end{align*}
    \textit{This contradiction shows no such integer $x$ exists, validating the proposition.}
\end{proof}



% Proposition 2.11
\section*{Proposition 2.11:}
Let $m \in \mathbb{N}$ and $n \in \mathbb{Z}$. If $mn \in \mathbb{N}$, then $n \in \mathbb{N}$.
\begin{proof}
    \textit{Assume $m \in \mathbb{N}$ and $mn \in \mathbb{N}$ for some $n \in \mathbb{Z}$.}
    \begin{align*}
        & \text{Since $m > 0$ and $mn > 0$, $n$ must be non-negative.} & & \text{(Definition of $\mathbb{N}$)} \\
        & \text{If $n$ were negative, $mn$ would be negative, contradicting $mn \in \mathbb{N}$.} & & \text{(Axiom 2.1(i))} \\
        & \text{Thus, $n$ must be non-negative and, since it's an integer, $n \in \mathbb{N}$.} & &
    \end{align*}
    \textit{Hence, we establish that $n \in \mathbb{N}$.}
\end{proof}



% Proposition 2.12
\section*{Proposition 2.12:}
For all $m,n,p \in \mathbb{Z}$:
\begin{enumerate}[label=(\roman*)]
    \item $-m < -n$ if and only if $m > n$.
    \item If $p > 0$ and $mp < np$ then $m < n$.
    \item If $p < 0$ and $mp < np$ then $n < m$.
    \item If $m \leq m$ and $0 \leq p$ then $mp \leq np$.
\end{enumerate}
\section*{Proposition 2.12 (i)}
\textit{For all $m,n \in \mathbb{Z}$, $-m < -n$ if and only if $m > n$.}
\begin{proof}
    \textit{We prove the bidirectional implication:}
    \begin{align*}
        & \text{($\Rightarrow$) Assume $-m < -n$. Multiplying both sides by $-1$ reverses the inequality: $m > n$.} & & \text{(Axiom 1.1(ii))} \\
        & \text{($\Leftarrow$) Assume $m > n$. Multiplying both sides by $-1$ gives: $-m < -n$.} & & \text{(Axiom 1.1(ii))} \\
    \end{align*}
    \textit{Thus, we establish the bi-conditional relationship.}
\end{proof}
\section*{Proposition 2.12 (ii)}
\textit{If $p > 0$ and $mp < np$ then $m < n$.}
\begin{proof}
    \textit{Given $p > 0$ and $mp < np$:}
    \begin{align*}
        & \text{Divide both sides by $p$ to obtain $m < n$.} & & \text{(Axiom 1.1(ii) and Definition of Division)} \\
    \end{align*}
    \textit{This division is valid as $p$ is positive, ensuring the inequality direction remains.}
\end{proof}
\section*{Proposition 2.12 (iii)}
\textit{If $p < 0$ and $mp < np$ then $n < m$.}
\begin{proof}
    \textit{Given $p < 0$ and $mp < np$:}
    \begin{align*}
        & \text{Multiplying by a negative number, we reverse the inequality: Divide by $p$ to get $n > m$.} & & \text{(Axiom 1.1(ii) and Definition of Division)} \\
    \end{align*}
    \textit{The direction of inequality changes due to multiplication by a negative number.}
\end{proof}
\section*{Proposition 2.12 (iv)}
\textit{If $m \leq n$ and $0 \leq p$ then $mp \leq np$.}
\begin{proof}
    \textit{Assuming $m \leq n$ and $0 \leq p$:}
    \begin{align*}
        & \text{Multiplication by a non-negative number preserves the inequality: $mp \leq np$.} & & \text{(Axiom 2.1(ii))} \\
    \end{align*}
    \textit{Since $p \geq 0$, the order relation remains consistent.}
\end{proof}


% Proposition 2.13
\section*{Proposition 2.13:}
$\mathbb{N} = \{ n \in \mathbb{Z} : n > 0 \}$.
\begin{proof}
    \textit{This proposition defines $\mathbb{N}$ based on the properties of integers:}
    \begin{align*}
        & \text{By the definition of natural numbers, each $n \in \mathbb{N}$ is greater than $0$.} & & \text{(Definition of $\mathbb{N}$)} \\
    \end{align*}
    \textit{Conversely, any positive integer by this definition is in $\mathbb{N}$.}
\end{proof}



% Proposition 2.14
\section*{Proposition 2.14:}
\begin{enumerate}[label=(\roman*)]
    \item $1 \in \mathbb{N}$.
    \item If $n \in \mathbb{N}$, then $n+1 \in \mathbb{N}$.
\end{enumerate}
\section*{Proposition 2.14 (i)}
\textit{$1 \in \mathbb{N}$.}
\begin{proof}
    \textit{This is a direct application of the definition of natural numbers:}
    \begin{align*}
        & \text{By the foundational definition, $1$ is included in $\mathbb{N}$.} & & \text{(Definition of $\mathbb{N}$)}
    \end{align*}
    \textit{Hence, it is established that $1 \in \mathbb{N}$.}
\end{proof}

\section*{Proposition 2.14 (ii)}
\textit{If $n \in \mathbb{N}$, then $n+1 \in \mathbb{N}$.}
\begin{proof}
    \textit{Assume $n \in \mathbb{N}$:}
    \begin{align*}
        & \text{By the properties of natural numbers, adding $1$ to any $n \in \mathbb{N}$ remains in $\mathbb{N}$.} & & \text{(Closure under addition, Axiom 2.1(i))} \\
    \end{align*}
    \textit{Therefore, we confirm that $n+1 \in \mathbb{N}$.}
\end{proof}



% Axiom 2.15 (Induction Axiom)
\section*{Axiom 2.15 (Induction Axiom):}
If a subset $A \subseteq \mathbb{Z}$ satisfies:
\begin{enumerate}
    \item $1 \in A$, and
    \item If $n \in A$, then $n+1 \in A$,
\end{enumerate}
then $\mathbb{N} \subseteq A$.



% Proposition 2.16
\section*{Proposition 2.16:}
Let $B \subseteq \mathbb{Z}$ be such that:
\begin{enumerate}
    \item $1 \in B$, and
    \item If $n \in B$, then $n+1 \in B$,
\end{enumerate}
then $B = \mathbb{N}$.

% Proposition 2.18
\section*{Proposition 2.18:}
\begin{enumerate}[label=(\roman*)]
    \item For all $k \in \mathbb{N}$, $k^3+2k$ is divisible by $3$.
    \item For all $k \in \mathbb{N}$, $k^4-6k^3+11k^2-6k$ is divisible by $4$.
    \item For all $k \in \mathbb{N}$, $k^3+5k$ is divisible by $6$.
\end{enumerate}

% Proposition 2.20
\section*{Proposition 2.20:}
For all $k \in \mathbb{N}$, $k \geq 1$.

% Proposition 2.21
\section*{Proposition 2.21:}
There exists no integer $x$ such that $0 < x < 1$.

% Corollary 2.22
\section*{Corollary 2.22:}
Let $n \in \mathbb{Z}$. There exists no integer $x$ such that $n < x < n+1$.

% Proposition 2.23
\section*{Proposition 2.23:}
Let $m,n \in \mathbb{N}$. If $n$ is divisible by $m$, then $m \leq n$.

% Proposition 2.24
\section*{Proposition 2.24:}
For all $k \in \mathbb{N}$, $k^2 + 1 > k$.

% Proposition 2.26
\section*{Proposition 2.26:}
For all integers $k \geq -3$, $3k^2 + 21k + 37 \geq 0$.

% Proposition 2.27
\section*{Proposition 2.27:}
For all integers $k \geq 2$, $k^2 < k^3$.

% Proposition 2.33
\section*{Proposition 2.33:}
Let $A$ be a nonempty subset of $\mathbb{Z}$ and $b \in \mathbb{Z}$, such that for each $a \in A$, $b \leq a$. Then $A$ has a smallest element.

% Proposition 2.34
\section*{Proposition 2.34:}
If $m$ and $n$ are integers that are not both $0$, then
\[ S = \{ k \in \mathbb{N} : k = mx + ny \text{ for some } x,y \in \mathbb{Z} \} \]
