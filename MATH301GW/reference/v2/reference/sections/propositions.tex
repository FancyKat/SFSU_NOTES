\subsection{Proposition 1.1}
Let \(X\), \(Y\), and \(Z\) be elements of the set of integers \(\mathbb{Z}\). This proposition demonstrates several foundational algebraic properties of integers:

% Enumeration for parts of the proposition
\begin{enumerate}[label=\thesubsection\ part \roman*]
    \item \label{prop:1.1-i} Commutativity of addition: \(X + Y = Y + X\).
    \item \label{prop:1.1-ii} Associativity of addition: \((X + Y) + Z = X + (Y + Z)\).
    \item \label{prop:1.1-iii} Existence of additive identity: \(X + 0 = X\).
    \item \label{prop:1.1-iv} Existence of additive inverse: For every \(X\), there exists an integer \(-X\) such that \(X + (-X) = 0\).
\end{enumerate}

\subsection{Proof of Proposition 1.1}
\textit{We now prove each part of Proposition 1.1:}

\begin{proof}[Proof of part (\ref{prop:1.1-i})]
We prove the commutativity of addition. Let \(X, Y \in \mathbb{Z}\). Consider:

% Aligned proof steps with step numbers on the left and equations centered
\begin{flalign*}
    &\text{(1)} & X + Y &= Y + X &\quad& \text{by the definition of commutative property in } \mathbb{Z}
\end{flalign*}
This concludes the proof of commutativity of addition.
\end{proof}

\begin{proof}[Proof of part (\ref{prop:1.1-ii})]
We prove the associativity of addition. Let \(X, Y, Z \in \mathbb{Z}\). Consider:

% Aligned proof steps with step numbers on the left and equations centered
\begin{flalign*}
    &\text{(1)} & (X + Y) + Z &= X + (Y + Z) &\quad& \text{by the definition of associative property in } \mathbb{Z}
\end{flalign*}
This concludes the proof of associativity of addition.
\end{proof}

% Continue similarly for parts (iii) and (iv)...
