% Theorem 4.4
\section*{Theorem 4.4:}
A legitimate method of describing a sequence $(y_j)_{j=m}^{\infty}$ is:
\begin{enumerate}
    \item to name $y_m$, and
    \item to state a formula describing $y_{n+1}$ in terms of $y_n$, for each $n \geq m$.
\end{enumerate}

% Proposition 4.5
\section*{Proposition 4.5:}
For all $k \in \mathbb{Z}_{\geq 0}$, $k! \in \mathbb{N}$.

% Proposition 4.6
\section*{Proposition 4.6:}
Let $b \in \mathbb{Z}$ and $k,m \in \mathbb{Z}_{\geq 0}$.
\begin{enumerate}
    \item if $b \in \mathbb{N}$ then $b^k \in \mathbb{N}$
    \item $b^mb^k = b^{m+k}$
    \item $(b^m)^k = b^{mk}$
\end{enumerate}

% Proposition 4.7
\section*{Proposition 4.7:}
For all $k \in \mathbb{N}$:
\begin{enumerate}
    \item $5^{2k-1}$ is divisible by $24$
    \item $2^{2k+1} + 1$ is divisible by $3$
    \item $10^{k+3} \cdot 4^{k+2} + 5$ is divisible by $9$
\end{enumerate}

% Proposition 4.8
\section*{Proposition 4.8:}
For all $k \in \mathbb{N}$, $4k > k$.

% Proposition 4.11
\section*{Proposition 4.11:}
Let $k \in \mathbb{N}$:
\begin{enumerate}
    \item $\sum_{j=1}^k j = \frac{k(k+1)}{2}$
    \item $\sum_{j=1}^k j^2 = \frac{k(k+1)(2k+1)}{6}$
\end{enumerate}

% Proposition 4.13
\section*{Proposition 4.13:}
For $x \neq 1$ and $k \in \mathbb{Z}_{\geq 0}$,
\[ \sum_{j=0}^k x^j = \frac{1 - x^{k+1}}{1 - x} \]

% Proposition 4.15
\section*{Proposition 4.15:}
\begin{enumerate}
    \item Let $m \in \mathbb{Z}$ and let $(x_j)_{j=1}^{\infty}$ be a sequence in $\mathbb{Z}$. Then for all $k \in \mathbb{N}$:
    \[ m \cdot \sum_{j=1}^k x_j! = \sum_{j=1}^k (mx_j) \]
    \item If $x_j = 1$ for all $j \in \mathbb{N}$, then for all $k \in \mathbb{N}$:
    \[ \sum_{j=1}^k x_j = k \]
    \item If $x_j = n \in \mathbb{Z}$ for all $j \in \mathbb{N}$, then for all $k \in \mathbb{N}$:
    \[ \sum_{j=1}^k x_j = kn \]
\end{enumerate}


% Proposition 4.16
\section*{Proposition 4.16:}
Let $(x_j)_{j=1}^{\infty}$ and $(y_j)_{j=1}^{\infty}$ be sequences in $\mathbb{Z}$, and let $a,b,c \in \mathbb{Z}$ be such that $a \leq b < c$.
\begin{enumerate}
    \item $\displaystyle\sum_{j=a}^{c} x_j = \displaystyle\sum_{j=a}^{b} x_j + \displaystyle\sum_{j=b+1}^{c} x_j$
    \item $\displaystyle\sum_{j=a}^{b} (x_j + y_j) = \displaystyle\sum_{j=a}^{b} x_j! + \displaystyle\sum_{j=a}^{b} y_j!$
\end{enumerate}

% Proposition 4.17
\section*{Proposition 4.17:}
Let $(x_j)_{j=1}^{\infty}$ be a sequence in $\mathbb{Z}$, and let $a,b, r \in \mathbb{Z}$ be such that $a \leq b$. Then
$\displaystyle\sum_{j=a}^{b} x_j = \displaystyle\sum_{j=a}^{b+r} x_j - r$

% Proposition 4.18
\section*{Proposition 4.18:}
Let $(x_j)_{j=1}^{\infty}$ and $(y_j)_{j=1}^{\infty}$ be sequences in $\mathbb{Z}$ such that $x_j \leq y_j$ for all $j \in \mathbb{N}$. Then for all $k \in \mathbb{N}$,
$\displaystyle\sum_{j=1}^{k} x_j \leq \displaystyle\sum_{j=1}^{k} y_j$

% Theorem 4.19
\section*{Theorem 4.19:}
Let $k,m \in \mathbb{Z}_{\geq 0}$, where $m \leq k$. Then $m!(k-m)!$ divides $k!$.

% Corollary 4.20
\section*{Corollary 4.20:}
For $1 \leq m \leq k$,
$\binom{k+1}{m} = \binom{k}{m-1} + \binom{k}{m}$

% Theorem 4.21
\section*{Theorem 4.21 (Binomial theorem for integers):}
If $a,b \in \mathbb{Z}$ and $k \in \mathbb{Z}_{\geq 0}$ then
$(a+b)^k = \displaystyle\sum_{m=0}^{k} \binom{k}{m} a^{k-m} b^m$

% Corollary 4.22
\section*{Corollary 4.22:}
For $k \in \mathbb{Z}_{\geq 0}$,
$\displaystyle\sum_{m=0}^{k} \binom{k}{m} = 2^k$

% Theorem 4.24
\section*{Theorem 4.24 (Principle of mathematical induction —second form):}
Let $P(k)$ be a statement depending on a variable $k \in \mathbb{N}$. In order to prove the statement ”$P(k)$ is true for all $k \in \mathbb{N}$” it is sufficient to prove:
\begin{enumerate}
    \item $P(1)$ is true and
    \item if $P(j)$ is true for all integers $j$ such that $1 \leq j \leq n$, then $P(n+1)$ is true
\end{enumerate}

% Proposition 4.29
\section*{Proposition 4.29:}
The $k$th Fibonacci number is given directly by the formula
$f_k = \frac{1}{\sqrt{5}} \left(\left(\frac{1 + \sqrt{5}}{2}\right)^k - \left(\frac{1 - \sqrt{5}}{2}\right)^k\right)$


% Proposition 4.30
\section*{Proposition 4.30:}
For all $k,m \in \mathbb{N}$, where $m \geq 2$,
\[ f_{m+k} = f_{m-1} f_k + f_m f_{k+1} \]

% Proposition 4.31
\section*{Proposition 4.31:}
For all $k \in \mathbb{N}$,
\[ f_{2k+1} = f_k^2 + f_{k+1}^2 \]

% Proposition 4.32
\section*{Proposition 4.32:}
For all $k,m \in \mathbb{N}$, $f_{mk}$ is divisible by $f_m$.
