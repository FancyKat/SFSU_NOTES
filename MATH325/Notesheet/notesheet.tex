
\documentclass[8pt, a4paper, landscape]{extarticle}
\usepackage[margin=0.15in]{geometry}
\usepackage{amsmath, amssymb, amsthm}
\usepackage{multicol}
\usepackage{graphicx}
\usepackage{enumerate}
\usepackage{tikz}
\usepackage{longtable}  % Add this line


\newcommand{\highlight}[1]{\textbf{\color{blue} #1}}
\setlength{\parindent}{0pt}
\setlength{\columnsep}{8pt}

\begin{document}

\begin{multicols*}{5}

  % Definitions and General Principles section
  \section*{Definitions}

  \textbf{Inverse of a Matrix}
  \begin{itemize}
    \item $(A^{-1})^{-1} = A$
    \item $(AB)^{-1} = B^{-1}A^{-1}$
    \item $AA^{-1} = A^{-1}A = I$
  \end{itemize}

  \textbf{LDU Decomposition}
  \begin{itemize}
    \item For a symmetric matrix 
    \[
      A: A = LDL^T
    \]
    \item $L$: lower triangular with unit diagonal
    \item $D$: diagonal matrix
  \end{itemize}

  \textbf{Vector Space Axioms}
  \begin{itemize}
    \item Addition: commutativity, associativity, identity, inverses
    \item Scalar Multiplication: distributivity, compatibility, identity
  \end{itemize}

  \textbf{Subspaces}
  \begin{itemize}
    \item Closed under addition and scalar multiplication
  \end{itemize}

  \textbf{Linear Dependence and Independence}
  \begin{itemize}
    \item Dependent: $\exists$ scalars, not all zero, s.t. $a_1v_1 + \ldots + a_nv_n = 0$
    \item Independent: only solution is $a_1 = \ldots = a_n = 0$
  \end{itemize}

  \textbf{Basis and Dimension}
  \begin{itemize}
    \item Basis: linearly independent spanning set
    \item Dimension: number of vectors in a basis
  \end{itemize}

  \textbf{General Principles for Subspaces}
  \begin{itemize}
    \item Closed under vector addition
    \item Closed under scalar multiplication
  \end{itemize}

  \textbf{Linear Transformation}
  \begin{itemize}
    \item Preserves vector addition and scalar multiplication
  \end{itemize}

  \textbf{Image and Kernel}
  \begin{itemize}
    \item $\text{im}(A)$: span of column vectors of $A$
    \item $\text{ker}(A)$: $\{x \in \mathbb{R}^n : Ax = 0\}$
  \end{itemize}

  \textbf{Basis Transformation}
  \begin{itemize}
    \item Unique representation of a vector in terms of basis vectors
  \end{itemize}

  \columnbreak

  \textbf{Determining Linear Independence (Standard Case)}

  Given vectors $v_1 = (1, 2, 3)$, $v_2 = (0, 1, 1)$, and $v_3 = (2, 5, 7)$, determine if they are linearly independent.

  \textbf{Solution:}
  \begin{enumerate}
    \item Arrange the vectors as columns in a matrix $A$:
          \[
            A = \begin{bmatrix}
              1 & 0 & 2 \\
              2 & 1 & 5 \\
              3 & 1 & 7
            \end{bmatrix}
          \]
    \item Perform row reduction on $A$:
          \[
            \begin{bmatrix}
              1 & 0 & 2 \\
              0 & 1 & 1 \\
              0 & 0 & 1
            \end{bmatrix}
          \]
    \item Since there are no rows of all zeros in the reduced row echelon form, the vectors are linearly independent.
  \end{enumerate}

  \textbf{Determining Linear Independence (Linearly Dependent Case)}

  Given vectors $v_1 = (1, 2, 3)$, $v_2 = (2, 4, 6)$, and $v_3 = (3, 6, 9)$, determine if they are linearly independent.

  \textbf{Solution:}
  \begin{enumerate}
    \item Arrange the vectors as columns in a matrix $A$:
          \[
            A = \begin{bmatrix}
              1 & 2 & 3 \\
              2 & 4 & 6 \\
              3 & 6 & 9
            \end{bmatrix}
          \]
    \item Perform row reduction on $A$:
          \[
            \begin{bmatrix}
              1 & 2 & 3 \\
              0 & 0 & 0 \\
              0 & 0 & 0
            \end{bmatrix}
          \]
    \item The presence of rows of all zeros indicates that the vectors are linearly dependent.
  \end{enumerate}

  \textbf{Finding a Basis for a Subspace (Polynomial Space)}

  Find a basis for the subspace of $P_3$ consisting of polynomials $p(x) = ax^3 + bx^2 + cx + d$ such that $p(1) = 0$.

  \textbf{Solution:}
  \begin{enumerate}
    \item The condition $p(1) = 0$ gives $a + b + c + d = 0$. To find a basis, express this condition in terms of the coefficients and set up a system.
    \item Considering the standard basis $\{1, x, x^2, x^3\}$ for $P_3$, impose the condition for $p(1) = 0$:
          \[
            \begin{bmatrix}
              1 & 1 & 1 & 1
            \end{bmatrix}
            \begin{bmatrix}
              d \\
              c \\
              b \\
              a
            \end{bmatrix}
            = 0
          \]
          This implies $a = -b - c - d$.
    \item A basis satisfying this condition is $\{x^3 - x^2, x^2 - x, x - 1\}$ as these polynomials nullify at $x = 1$ and are linearly independent.
  \end{enumerate}

  \textbf{Finding the Matrix Inverse}

  Find the inverse of the matrix $A = \begin{bmatrix} 1 & 3 \\ 2 & 7 \end{bmatrix}$.

  \textbf{Solution:}
  \begin{enumerate}
    \item Set up the augmented matrix for $A$ and the identity matrix: $\begin{bmatrix} 1 & 3 & | & 1 & 0 \\ 2 & 7 & | & 0 & 1 \end{bmatrix}$.
    \item Perform row operations to get the identity matrix on the left side of the augmented matrix. Subtract twice the first row from the second row to start: $\begin{bmatrix} 1 & 3 & | & 1 & 0 \\ 0 & 1 & | & -2 & 1 \end{bmatrix}$.
    \item Then, subtract 3 times the second row from the first row: $\begin{bmatrix} 1 & 0 & | & 7 & -3 \\ 0 & 1 & | & -2 & 1 \end{bmatrix}$.
    \item The matrix on the right side is now $A^{-1} = \begin{bmatrix} 7 & -3 \\ -2 & 1 \end{bmatrix}$.
  \end{enumerate}

  \textbf{Eigenvalues and Eigenvectors}

  Find the eigenvalues and corresponding eigenvectors for the matrix $B = \begin{bmatrix} 4 & 1 \\ 1 & 4 \end{bmatrix}$.

  \textbf{Solution:}
  \begin{enumerate}
    \item Find the characteristic polynomial: $\det(B - \lambda I) = \det \begin{bmatrix} 4-\lambda & 1 \\ 1 & 4-\lambda \end{bmatrix} = (4-\lambda)^2 - 1$.
    \item Solve for $\lambda$: $(4-\lambda)^2 - 1 = 0 \Rightarrow \lambda^2 - 8\lambda + 15 = 0$. The eigenvalues are $\lambda_1 = 3$ and $\lambda_2 = 5$.
    \item Find eigenvectors for each eigenvalue:
          - For $\lambda_1 = 3$: Solve $(B - 3I)x = 0$. This gives $x_1 = \begin{bmatrix} 1 \\ -1 \end{bmatrix}$.
          - For $\lambda_2 = 5$: Solve $(B - 5I)x = 0$. This gives $x_2 = \begin{bmatrix} 1 \\ 1 \end{bmatrix}$.
  \end{enumerate}

  \textbf{Diagonalization}

  Determine if the matrix $C = \begin{bmatrix} 2 & 1 \\ 0 & 2 \end{bmatrix}$ is diagonalizable. If it is, find a matrix $P$ that diagonalizes $C$.

  \textbf{Solution:}
  \begin{enumerate}
    \item Find the eigenvalues of $C$: The characteristic polynomial is $(2 - \lambda)^2 = 0$, so the only eigenvalue is $\lambda = 2$.
    \item Since $C$ is a $2 \times 2$ matrix with only one distinct eigenvalue, we need to check if there are two linearly independent eigenvectors corresponding to $\lambda = 2$.
    \item Solve $(C - 2I)x = 0$: This leads to the system $x_2 = 0$, indicating that every eigenvector has the form $\begin{bmatrix} t \\ 0 \end{bmatrix}$, which does not provide two independent eigenvectors.
    \item Since we cannot find two linearly independent eigenvectors, $C$ is not diagonalizable.
  \end{enumerate}

  % Practice Problem: Vector Space Axioms
  \textbf{Verifying Vector Space Axioms}

  Verify that the set of all polynomials of degree at most 2 with real coefficients forms a vector space over the real numbers.

  \textbf{Solution}

  To verify that the set of all polynomials of degree at most 2 with real coefficients forms a vector space, we need to check that the following axioms hold:
  \begin{itemize}
    \item The set is closed under addition: The sum of any two polynomials of degree at most 2 is also a polynomial of degree at most 2.
    \item The set is closed under scalar multiplication: The scalar multiple of any polynomial of degree at most 2 is also a polynomial of degree at most 2.
    \item The set contains a zero vector, which is the zero polynomial.
    \item Each polynomial has an additive inverse within the set.
    \item Addition is associative and commutative.
    \item Scalar multiplication is distributive with respect to both scalar and vector addition.
    \item Scalar multiplication is compatible with field multiplication.
    \item The scalar 1 acts as a multiplicative identity.
  \end{itemize}
  Since all these properties are satisfied, the set is indeed a vector space.


  \textbf{Finding the Rank of a Matrix}

  Find the rank of the matrix
  \[
    E = \begin{bmatrix} 1 & 2 & 3 \\ 2 & 4 & 6 \\ 3 & 6 & 9 \end{bmatrix}
  \]

  \textbf{Solution:}
  \begin{enumerate}
    \item Perform row reduction on $E$:
          \[
            \begin{bmatrix}
              1 & 2 & 3 \\
              0 & 0 & 0 \\
              0 & 0 & 0
            \end{bmatrix}.
          \]
    \item The rank is equal to the number of non-zero rows in the reduced row echelon form. Hence, $\text{rank}(E) = 1$.
  \end{enumerate}

  \textbf{Linear Dependence in $\mathbb{R}^2$}

  Given the vectors $v_1 = (3, -1)$ and $v_2 = (6, -2)$ in $\mathbb{R}^2$, determine if $v_1$ and $v_2$ are linearly dependent.
  \textbf{Solution}
  \begin{enumerate}
    \item Notice that $v_2 = 2v_1$, which means $v_2$ is a scalar multiple of $v_1$.
    \item This implies the set $\{v_1, v_2\}$ is linearly dependent because $v_2$ can be expressed as a linear combination of $v_1$.
  \end{enumerate}

  \textbf{Identifying Subspaces}

  Determine if the set of all vectors in $\mathbb{R}^3$ of the form $(a, a, b)$ is a subspace of $\mathbb{R}^3$.
    
  \textbf{Solution}

  To determine if the set of all vectors in $\mathbb{R}^3$ of the form $(a, a, b)$ is a subspace, we check for the following properties:
  \begin{itemize}
    \item The zero vector $(0, 0, 0)$ is in the set, satisfying the non-emptiness requirement.
    \item The set is closed under addition: $(a_1, a_1, b_1) + (a_2, a_2, b_2) = (a_1 + a_2, a_1 + a_2, b_1 + b_2)$, which is of the form $(a, a, b)$.
    \item The set is closed under scalar multiplication: For any real number $c$, $c(a, a, b) = (ca, ca, cb)$, which is still of the form $(a, a, b)$.
  \end{itemize}
  Thus, the set is a subspace of $\mathbb{R}^3$.

  \textbf{Important Identities and Properties}
  \begin{itemize}
    \item Rank-Nullity Theorem: $\text{rank}(A) + \text{nullity}(A) = n$, where $n$ is the number of columns of $A$.
    \item Determinant properties: $\det(AB) = \det(A)\det(B)$ and $\det(A^{-1}) = \frac{1}{\det(A)}$ for invertible $A$.
  \end{itemize}

\end{multicols*}

\pagebreak

\begin{multicols*}{4}


  \textbf{Determinant Calculation}

  Calculate the determinant of the matrix
  \[
    D = \begin{bmatrix} 6 & 1 & 2 \\ 1 & 3 & 1 \\ 2 & 0 & 4 \end{bmatrix}
  \]

  \textbf{Solution:}
  \begin{enumerate}
    \item Apply the Laplace expansion using the first row:
          \[
            \det(D) = 6\begin{vmatrix} 3 & 1 \\ 0 & 4 \end{vmatrix} - 1\begin{vmatrix} 1 & 1 \\ 2 & 4 \end{vmatrix} + 2\begin{vmatrix} 1 & 3 \\ 2 & 0 \end{vmatrix}
          \]
    \item Calculate each minor:
          \begin{equation}
            \begin{aligned}
               & 6\cdot(3 \cdot 4 - 0 \cdot 1) - 1\cdot(1 \cdot 4 - 1 \cdot 2) + 2\cdot(1 \cdot 0 - 3 \cdot 2) \\
               & = 72 - 2 - 12                                                                                 \\
               & = 58
            \end{aligned}
          \end{equation}
    \item Thus, $\det(D) = 58$.
  \end{enumerate}

  \textbf{Using Cramer's Rule}

  Solve the following system of equations using Cramer's Rule:
  \begin{equation}
    \begin{aligned}
      x + 2y - z  & = 4,  \\
      2x - y + 3z & = -2, \\
      x + 3y + z  & = 3.
    \end{aligned}
  \end{equation}
  \textbf{Solution:}
  \begin{enumerate}
    \item Write the coefficient matrix and calculate its determinant:
          \[
            A = \begin{bmatrix}
              1 & 2  & -1 \\
              2 & -1 & 3  \\
              1 & 3  & 1
            \end{bmatrix}, \quad \det(A) = -16.
          \]
    \item For $x$, replace the first column of $A$ with the constant terms and calculate its determinant:
          \[
            A_x = \begin{bmatrix}
              4  & 2  & -1 \\
              -2 & -1 & 3  \\
              3  & 3  & 1
            \end{bmatrix}, \quad \det(A_x) = -16.
          \]
    \item For $y$, replace the second column of $A$:
          \[
            A_y = \begin{bmatrix}
              1 & 4  & -1 \\
              2 & -2 & 3  \\
              1 & 3  & 1
            \end{bmatrix}, \quad \det(A_y) = -32.
          \]
    \item For $z$, replace the third column of $A$:
          \[
            A_z = \begin{bmatrix}
              1 & 2  & 4  \\
              2 & -1 & -2 \\
              1 & 3  & 3
            \end{bmatrix}, \quad \det(A_z) = -16.
          \]
    \item Compute the solutions: $x = \frac{\det(A_x)}{\det(A)} = 1, y = \frac{\det(A_y)}{\det(A)} = 2, z = \frac{\det(A_z)}{\det(A)} = 1$.
  \end{enumerate}



  % Problem 10: LU Decomposition
  \textbf{LU Decomposition}

  Perform LU decomposition on the matrix $F = \begin{bmatrix} 4 & 3 \\ 6 & 3 \end{bmatrix}$.

  \textbf{Solution:}
  \begin{enumerate}
    \item Express $F$ as the product of a lower triangular matrix $L$ and an upper triangular matrix $U$.
    \item Choose $L$ with 1s on the diagonal: $L = \begin{bmatrix} 1 & 0 \\ l_{21} & 1 \end{bmatrix}$.
    \item Let $U = \begin{bmatrix} u_{11} & u_{12} \\ 0 & u_{22} \end{bmatrix}$.
    \item Since $F_{21} = l_{21} \cdot u_{11}$ and $F_{21} = 6$, $u_{11} = 4$, we get $l_{21} = \frac{6}{4} = \frac{3}{2}$.
    \item Solve for $U$ using the first row of $F$: $U = \begin{bmatrix} 4 & 3 \\ 0 & u_{22} \end{bmatrix}$. The second element of the second row gives $u_{22} = 3 - \frac{3}{2} \cdot 3 = -\frac{3}{2}$.
    \item The LU decomposition is $L = \begin{bmatrix} 1 & 0 \\ \frac{3}{2} & 1 \end{bmatrix}$, $U = \begin{bmatrix} 4 & 3 \\ 0 & -\frac{3}{2} \end{bmatrix}$
  \end{enumerate}

  \textbf{Orthogonal Diagonalization}

  Orthogonally diagonalize the matrix
  \[
    F = \begin{bmatrix} 3 & -1 \\ -1 & 3 \end{bmatrix}
  \]

  \textbf{Solution:}
  \begin{enumerate}
    \item Find the eigenvalues by solving $\det(F - \lambda I) = 0$: $\lambda^2 - 6\lambda + 8 = 0$ gives $\lambda_1 = 2$ and $\lambda_2 = 4$.
    \item Find the eigenvectors: For $\lambda_1 = 2$, solve $(F - 2I)x = 0$ to get $x_1 = \begin{bmatrix} 1 \\ 1 \end{bmatrix}$ (after normalization). For $\lambda_2 = 4$, solve $(F - 4I)x = 0$ to get $x_2 = \begin{bmatrix} 1 \\ -1 \end{bmatrix}$ (after normalization).
    \item Construct $P = \begin{bmatrix} 1/\sqrt{2} & 1/\sqrt{2} \\ 1/\sqrt{2} & -1/\sqrt{2} \end{bmatrix}$ and verify $P^T F P = \begin{bmatrix} 2 & 0 \\ 0 & 4 \end{bmatrix}$.
  \end{enumerate}


  \textbf{Orthogonal Diagonalization}

  Orthogonally diagonalize the matrix $F = \begin{bmatrix} 3 & -1 \\ -1 & 3 \end{bmatrix}$.

  \textbf{Solution:}
  \begin{enumerate}
    \item Find the eigenvalues by solving $\det(F - \lambda I) = 0$: $\lambda^2 - 6\lambda + 8 = 0$ gives $\lambda_1 = 2$ and $\lambda_2 = 4$.
    \item Find the eigenvectors: For $\lambda_1 = 2$, solve $(F - 2I)x = 0$ to get $x_1 = \begin{bmatrix} 1 \\ 1 \end{bmatrix}$ (after normalization). For $\lambda_2 = 4$, solve $(F - 4I)x = 0$ to get $x_2 = \begin{bmatrix} 1 \\ -1 \end{bmatrix}$ (after normalization).
    \item Construct $P = \begin{bmatrix} 1/\sqrt{2} & 1/\sqrt{2} \\ 1/\sqrt{2} & -1/\sqrt{2} \end{bmatrix}$ and verify $P^T F P = \begin{bmatrix} 2 & 0 \\ 0 & 4 \end{bmatrix}$.
  \end{enumerate}


  \textbf{Basis and Dimension of a Vector Space}

  Consider the vector space $V$ of all vectors in $\mathbb{R}^4$ that satisfy the equation $x_1 - 2x_2 + x_3 - 2x_4 = 0$. Find a basis for $V$ and state its dimension.
  
  \textbf{Solution}

  \begin{enumerate}
    \item To find a basis, we need to solve for the vectors that satisfy the given equation. Let $x_4 = t$, then $x_3 = 2t$, $x_2 = s$, and $x_1 = 2s - t$.
    \item Thus, any vector in $V$ can be written as $\begin{bmatrix} 2s - t \\ s \\ 2t \\ t \end{bmatrix} = s\begin{bmatrix} 2 \\ 1 \\ 0 \\ 0 \end{bmatrix} + t\begin{bmatrix} -1 \\ 0 \\ 2 \\ 1 \end{bmatrix}$.
    \item The vectors $\begin{bmatrix} 2 \\ 1 \\ 0 \\ 0 \end{bmatrix}$ and $\begin{bmatrix} -1 \\ 0 \\ 2 \\ 1 \end{bmatrix}$ are linearly independent and span $V$, so they form a basis for $V$.
    \item The dimension of $V$, denoted as $\text{dim}(V)$, is the number of vectors in the basis, which is 2.
  \end{enumerate}

  \textbf{Span of Vectors}

  Determine if the vector $\mathbf{u} = \begin{bmatrix} 4 \\ -2 \\ 1 \end{bmatrix}$ is in the span of the set of vectors $\mathbf{v}_1 = \begin{bmatrix} 1 \\ 0 \\ -1 \end{bmatrix}$, $\mathbf{v}_2 = \begin{bmatrix} 0 \\ 1 \\ 2 \end{bmatrix}$.
  
  \textbf{Solution:}
  \begin{enumerate}
      \item We need to determine if there exist scalars $c_1$ and $c_2$ such that $\mathbf{u} = c_1\mathbf{v}_1 + c_2\mathbf{v}_2$.
      \item Setting up the equation, we get:
      \[
      \begin{bmatrix} 4 \\ -2 \\ 1 \end{bmatrix} = c_1\begin{bmatrix} 1 \\ 0 \\ -1 \end{bmatrix} + c_2\begin{bmatrix} 0 \\ 1 \\ 2 \end{bmatrix}
      \]
      \item This leads to the system of equations:
      \begin{align*}
          c_1 &= 4, \\
          c_2 &= -2, \\
          -c_1 + 2c_2 &= 1.
      \end{align*}
      \item Solving this system, we find $c_1 = 4$ and $c_2 = -2$. However, these values do not satisfy the third equation ($-4 - 2 \neq 1$). Therefore, $\mathbf{u}$ is not in the span of $\mathbf{v}_1$ and $\mathbf{v}_2$.
  \end{enumerate}
  
  \textbf{Change of Basis}

  Given a vector $v = \begin{bmatrix} 3 \\ 2 \end{bmatrix}$ in the standard basis of $\mathbb{R}^2$, find its coordinates in the new basis $B = \left\{ \begin{bmatrix} 1 \\ 1 \end{bmatrix}, \begin{bmatrix} -1 \\ 1 \end{bmatrix} \right\}$.
  \textbf{Solution}
  \begin{enumerate}
    \item The coordinates of $v$ in the basis $B$ can be found by solving the equation $c_1\begin{bmatrix} 1 \\ 1 \end{bmatrix} + c_2\begin{bmatrix} -1 \\ 1 \end{bmatrix} = \begin{bmatrix} 3 \\ 2 \end{bmatrix}$.
    \item This equation translates into the system:
          \[
            c_1 - c_2 = 3, \\
            c_1 + c_2 = 2.
          \]
    \item Adding the two equations yields $2c_1 = 5$, so $c_1 = \frac{5}{2}$.
    \item Substituting $c_1$ into the second equation gives $c_2 = \frac{2}{2} - \frac{5}{2} = -\frac{3}{2}$.
    \item Therefore, the coordinates of $v$ in the basis $B$ are $\left(\frac{5}{2}, -\frac{3}{2}\right)$.
  \end{enumerate}


  \textbf{Perform LDU decomposition}

  Perform LDU decomposition on the matrix
  \[
    H = \begin{bmatrix} 4 & 12 & -16 \\ 12 & 37 & -43 \\ -16 & -43 & 98 \end{bmatrix}
  \]
  \textbf{Solution:}
  \begin{enumerate}
    \item First, we find the matrix $L$ such that $H = LDU$ where $L$ is a lower triangular matrix with unit diagonal, $D$ is a diagonal matrix, and $U$ is an upper triangular matrix.
    \item Decompose $H$ into $LDU$:
          \[
            L = \begin{bmatrix}
              1  & 0 & 0 \\
              3  & 1 & 0 \\
              -4 & 5 & 1
            \end{bmatrix}, \quad
            D = \begin{bmatrix}
              4 & 0 & 0 \\
              0 & 1 & 0 \\
              0 & 0 & 9
            \end{bmatrix}, \quad
          \]
          \[
            U = \begin{bmatrix}
              1 & 3 & -4 \\
              0 & 1 & 5  \\
              0 & 0 & 1
            \end{bmatrix}.
          \]
    \item Verify the decomposition by calculating $LDU$ and comparing it with $H$:
          \[
            LDU = \begin{bmatrix}
              1  & 0 & 0 \\
              3  & 1 & 0 \\
              -4 & 5 & 1
            \end{bmatrix}
            \begin{bmatrix}
              4 & 0 & 0 \\
              0 & 1 & 0 \\
              0 & 0 & 9
            \end{bmatrix}
            \begin{bmatrix}
              1 & 3 & -4 \\
              0 & 1 & 5  \\
              0 & 0 & 1
            \end{bmatrix}
          \]
          \[
            = \begin{bmatrix}
              4   & 12  & -16 \\
              12  & 37  & -43 \\
              -16 & -43 & 98
            \end{bmatrix}.
          \]
    \item The result confirms the LDU decomposition of $H$.
  \end{enumerate}


\end{multicols*}


\end{document}