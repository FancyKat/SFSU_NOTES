
\documentclass[8pt, a4paper, landscape]{extarticle}
\usepackage[margin=0.12in]{geometry}
\usepackage{amsmath, amssymb, amsthm}
\usepackage{multicol}
\usepackage{graphicx}
\usepackage{enumerate}
\usepackage{tikz}
\usepackage{longtable}  % Add this line


\newcommand{\highlight}[1]{\textbf{\color{blue} #1}}
\setlength{\parindent}{0pt}
\setlength{\columnsep}{20pt}

\begin{document}

\begin{multicols*}{4}

  % Definitions and General Principles section
  \section*{Definitions and General Principles}

  \textbf{Inverse of a Matrix}
  \begin{itemize}
    \item $(A^{-1})^{-1} = A$
    \item $(AB)^{-1} = B^{-1}A^{-1}$
    \item $AA^{-1} = A^{-1}A = I$
  \end{itemize}

  \textbf{LDU Decomposition}
  \begin{itemize}
    \item For a symmetric matrix $A$: $A = LDL^T$
    \item $L$: lower triangular with unit diagonal
    \item $D$: diagonal matrix
  \end{itemize}

  \textbf{Vector Space Axioms}
  \begin{itemize}
    \item Addition: commutativity, associativity, identity, inverses
    \item Scalar Multiplication: distributivity, compatibility, identity
  \end{itemize}

  \textbf{Subspaces}
  \begin{itemize}
    \item Closed under addition and scalar multiplication
  \end{itemize}

  \textbf{Linear Dependence and Independence}
  \begin{itemize}
    \item Dependent: $\exists$ scalars, not all zero, s.t. $a_1v_1 + \ldots + a_nv_n = 0$
    \item Independent: only solution is $a_1 = \ldots = a_n = 0$
  \end{itemize}

  \textbf{Basis and Dimension}
  \begin{itemize}
    \item Basis: linearly independent spanning set
    \item Dimension: number of vectors in a basis
  \end{itemize}

  \textbf{General Principles for Subspaces}
  \begin{itemize}
    \item Closed under vector addition
    \item Closed under scalar multiplication
  \end{itemize}

  \textbf{Linear Transformation}
  \begin{itemize}
    \item Preserves vector addition and scalar multiplication
  \end{itemize}

  \textbf{Image and Kernel}
  \begin{itemize}
    \item $\text{im}(A)$: span of column vectors of $A$
    \item $\text{ker}(A)$: $\{x \in \mathbb{R}^n : Ax = 0\}$
  \end{itemize}

  \textbf{Basis Transformation}
  \begin{itemize}
    \item Unique representation of a vector in terms of basis vectors
  \end{itemize}

  \columnbreak

  % SECTION 1
  % Determining Linear Independence
  \textbf{Determining Linear Independence (Standard Case)}

  Given vectors $v_1 = (1, 2, 3)$, $v_2 = (0, 1, 1)$, and $v_3 = (2, 5, 7)$, determine if they are linearly independent.

  \textbf{Solution:}
  \begin{enumerate}
    \item Arrange the vectors as columns in a matrix $A$:
          \[
            A = \begin{bmatrix}
              1 & 0 & 2 \\
              2 & 1 & 5 \\
              3 & 1 & 7
            \end{bmatrix}
          \]
    \item Perform row reduction on $A$:
          \[
            \begin{bmatrix}
              1 & 0 & 2 \\
              0 & 1 & 1 \\
              0 & 0 & 1
            \end{bmatrix}
          \]
    \item Since there are no rows of all zeros in the reduced row echelon form, the vectors are linearly independent.
  \end{enumerate}

  % Problem 2: Linear Dependence
  \textbf{Determining Linear Independence (Linearly Dependent Case)}

  Given vectors $v_1 = (1, 2, 3)$, $v_2 = (2, 4, 6)$, and $v_3 = (3, 6, 9)$, determine if they are linearly independent.

  \textbf{Solution:}
  \begin{enumerate}
    \item Arrange the vectors as columns in a matrix $A$:
          \[
            A = \begin{bmatrix}
              1 & 2 & 3 \\
              2 & 4 & 6 \\
              3 & 6 & 9
            \end{bmatrix}
          \]
    \item Perform row reduction on $A$:
          \[
            \begin{bmatrix}
              1 & 2 & 3 \\
              0 & 0 & 0 \\
              0 & 0 & 0
            \end{bmatrix}
          \]
    \item The presence of rows of all zeros indicates that the vectors are linearly dependent.
  \end{enumerate}

  % Problem 3: Finding a Basis for a Subspace
  \textbf{Finding a Basis for a Subspace (Polynomial Space)}

  Find a basis for the subspace of $P_3$ consisting of polynomials $p(x) = ax^3 + bx^2 + cx + d$ such that $p(1) = 0$.

  \textbf{Solution:}
  \begin{enumerate}
    \item The condition $p(1) = 0$ gives $a + b + c + d = 0$. To find a basis, express this condition in terms of the coefficients and set up a system.
    \item Considering the standard basis $\{1, x, x^2, x^3\}$ for $P_3$, impose the condition for $p(1) = 0$:
          \[
            \begin{bmatrix}
              1 & 1 & 1 & 1
            \end{bmatrix}
            \begin{bmatrix}
              d \\
              c \\
              b \\
              a
            \end{bmatrix}
            = 0
          \]
          This implies $a = -b - c - d$.
    \item A basis satisfying this condition is $\{x^3 - x^2, x^2 - x, x - 1\}$ as these polynomials nullify at $x = 1$ and are linearly independent.
  \end{enumerate}

  \columnbreak

  % SECTION 2
  % Problem 4: Matrix Inverse
  \textbf{Finding the Matrix Inverse}

  Find the inverse of the matrix $A = \begin{bmatrix} 1 & 3 \\ 2 & 7 \end{bmatrix}$.

  \textbf{Solution:}
  \begin{enumerate}
    \item Set up the augmented matrix for $A$ and the identity matrix: $\begin{bmatrix} 1 & 3 & | & 1 & 0 \\ 2 & 7 & | & 0 & 1 \end{bmatrix}$.
    \item Perform row operations to get the identity matrix on the left side of the augmented matrix. Subtract twice the first row from the second row to start: $\begin{bmatrix} 1 & 3 & | & 1 & 0 \\ 0 & 1 & | & -2 & 1 \end{bmatrix}$.
    \item Then, subtract 3 times the second row from the first row: $\begin{bmatrix} 1 & 0 & | & 7 & -3 \\ 0 & 1 & | & -2 & 1 \end{bmatrix}$.
    \item The matrix on the right side is now $A^{-1} = \begin{bmatrix} 7 & -3 \\ -2 & 1 \end{bmatrix}$.
  \end{enumerate}

  % Problem 5: Eigenvalues and Eigenvectors
  \textbf{Eigenvalues and Eigenvectors}

  Find the eigenvalues and corresponding eigenvectors for the matrix $B = \begin{bmatrix} 4 & 1 \\ 1 & 4 \end{bmatrix}$.

  \textbf{Solution:}
  \begin{enumerate}
    \item Find the characteristic polynomial: $\det(B - \lambda I) = \det \begin{bmatrix} 4-\lambda & 1 \\ 1 & 4-\lambda \end{bmatrix} = (4-\lambda)^2 - 1$.
    \item Solve for $\lambda$: $(4-\lambda)^2 - 1 = 0 \Rightarrow \lambda^2 - 8\lambda + 15 = 0$. The eigenvalues are $\lambda_1 = 3$ and $\lambda_2 = 5$.
    \item Find eigenvectors for each eigenvalue:
          - For $\lambda_1 = 3$: Solve $(B - 3I)x = 0$. This gives $x_1 = \begin{bmatrix} 1 \\ -1 \end{bmatrix}$.
          - For $\lambda_2 = 5$: Solve $(B - 5I)x = 0$. This gives $x_2 = \begin{bmatrix} 1 \\ 1 \end{bmatrix}$.
  \end{enumerate}

  \textbf{Diagonalization}

  Determine if the matrix $C = \begin{bmatrix} 2 & 1 \\ 0 & 2 \end{bmatrix}$ is diagonalizable. If it is, find a matrix $P$ that diagonalizes $C$.

  \textbf{Solution:}
  \begin{enumerate}
    \item Find the eigenvalues of $C$: The characteristic polynomial is $(2 - \lambda)^2 = 0$, so the only eigenvalue is $\lambda = 2$.
    \item Since $C$ is a $2 \times 2$ matrix with only one distinct eigenvalue, we need to check if there are two linearly independent eigenvectors corresponding to $\lambda = 2$.
    \item Solve $(C - 2I)x = 0$: This leads to the system $x_2 = 0$, indicating that every eigenvector has the form $\begin{bmatrix} t \\ 0 \end{bmatrix}$, which does not provide two independent eigenvectors.
    \item Since we cannot find two linearly independent eigenvectors, $C$ is not diagonalizable.
  \end{enumerate}

  \columnbreak

  % SECTION 4
  \textbf{Perform LDU decomposition}

  Perform LDU decomposition on the matrix $H = \begin{bmatrix} 4 & 12 & -16 \\ 12 & 37 & -43 \\ -16 & -43 & 98 \end{bmatrix}$.

  \textbf{Solution:}
  \begin{enumerate}
    \item First, we find the matrix $L$ such that $H = LDU$ where $L$ is a lower triangular matrix with unit diagonal, $D$ is a diagonal matrix, and $U$ is an upper triangular matrix.
    \item Decompose $H$ into $LDU$:
          \[
            L = \begin{bmatrix}
              1  & 0 & 0 \\
              3  & 1 & 0 \\
              -4 & 5 & 1
            \end{bmatrix}, \quad
            D = \begin{bmatrix}
              4 & 0 & 0 \\
              0 & 1 & 0 \\
              0 & 0 & 9
            \end{bmatrix}, \quad
          \]
          \[
            U = \begin{bmatrix}
              1 & 3 & -4 \\
              0 & 1 & 5  \\
              0 & 0 & 1
            \end{bmatrix}.
          \]
    \item Verify the decomposition by calculating $LDU$ and comparing it with $H$:
          \[
            LDU = \begin{bmatrix}
              1  & 0 & 0 \\
              3  & 1 & 0 \\
              -4 & 5 & 1
            \end{bmatrix}
            \begin{bmatrix}
              4 & 0 & 0 \\
              0 & 1 & 0 \\
              0 & 0 & 9
            \end{bmatrix}
            \begin{bmatrix}
              1 & 3 & -4 \\
              0 & 1 & 5  \\
              0 & 0 & 1
            \end{bmatrix}
            = \begin{bmatrix}
              4   & 12  & -16 \\
              12  & 37  & -43 \\
              -16 & -43 & 98
            \end{bmatrix}.
          \]
    \item The result confirms the LDU decomposition of $H$.
  \end{enumerate}



\end{multicols*}

\pagebreak

\begin{multicols*}{3}

  % === COLUMN 5: Practice Problems and Solutions ===
  \section*{Section 3}

  % Problem 7: Determinant Calculation
  \textbf{Problem 7: Determinant Calculation}

  Calculate the determinant of the matrix $D = \begin{bmatrix} 6 & 1 & 2 \\ 1 & 3 & 1 \\ 2 & 0 & 4 \end{bmatrix}$.

  \textbf{Solution:}
  \begin{enumerate}
    \item Apply the Laplace expansion using the first row:
          \[
            \det(D) = 6\begin{vmatrix} 3 & 1 \\ 0 & 4 \end{vmatrix} - 1\begin{vmatrix} 1 & 1 \\ 2 & 4 \end{vmatrix} + 2\begin{vmatrix} 1 & 3 \\ 2 & 0 \end{vmatrix}
          \]
    \item Calculate each minor:
          \[
            6(3 \cdot 4 - 0 \cdot 1) - 1(1 \cdot 4 - 1 \cdot 2) + 2(1 \cdot 0 - 3 \cdot 2) = 72 - 2 - 12 = 58
          \]
    \item Thus, $\det(D) = 58$.
  \end{enumerate}

  % Problem 8: Cramer's Rule
  \textbf{Problem 8: Using Cramer's Rule}

  Solve the following system of equations using Cramer's Rule:


  \begin{equation}
    \begin{aligned}
      x + 2y - z  & = 4,  \\
      2x - y + 3z & = -2, \\
      x + 3y + z  & = 3.\end{aligned}
  \end{equation}

  \textbf{Solution:}
  \begin{enumerate}
    \item Write the coefficient matrix and calculate its determinant:
          \[
            A = \begin{bmatrix}
              1 & 2  & -1 \\
              2 & -1 & 3  \\
              1 & 3  & 1
            \end{bmatrix}, \quad \det(A) = -16.
          \]
    \item For $x$, replace the first column of $A$ with the constant terms and calculate its determinant:
          \[
            A_x = \begin{bmatrix}
              4  & 2  & -1 \\
              -2 & -1 & 3  \\
              3  & 3  & 1
            \end{bmatrix}, \quad \det(A_x) = -16.
          \]
    \item For $y$, replace the second column of $A$:
          \[
            A_y = \begin{bmatrix}
              1 & 4  & -1 \\
              2 & -2 & 3  \\
              1 & 3  & 1
            \end{bmatrix}, \quad \det(A_y) = -32.
          \]
    \item For $z$, replace the third column of $A$:
          \[
            A_z = \begin{bmatrix}
              1 & 2  & 4  \\
              2 & -1 & -2 \\
              1 & 3  & 3
            \end{bmatrix}, \quad \det(A_z) = -16.
          \]
    \item Compute the solutions: $x = \frac{\det(A_x)}{\det(A)} = 1, y = \frac{\det(A_y)}{\det(A)} = 2, z = \frac{\det(A_z)}{\det(A)} = 1$.
  \end{enumerate}


  \columnbreak

  \section*{Section 4}

  % Problem 9: Rank of a Matrix
  \textbf{Problem 9: Finding the Rank of a Matrix}

  Find the rank of the matrix $E = \begin{bmatrix} 1 & 2 & 3 \\ 2 & 4 & 6 \\ 3 & 6 & 9 \end{bmatrix}$.

  \textbf{Solution:}
  \begin{enumerate}
    \item Perform row reduction on $E$:
          \[
            \begin{bmatrix}
              1 & 2 & 3 \\
              0 & 0 & 0 \\
              0 & 0 & 0
            \end{bmatrix}.
          \]
    \item The rank is equal to the number of non-zero rows in the reduced row echelon form. Hence, $\text{rank}(E) = 1$.
  \end{enumerate}

  % Problem 10: LU Decomposition
  \textbf{Problem 10: LU Decomposition}

  Perform LU decomposition on the matrix $F = \begin{bmatrix} 4 & 3 \\ 6 & 3 \end{bmatrix}$.

  \textbf{Solution:}
  \begin{enumerate}
    \item Express $F$ as the product of a lower triangular matrix $L$ and an upper triangular matrix $U$.
    \item Choose $L$ with 1s on the diagonal: $L = \begin{bmatrix} 1 & 0 \\ l_{21} & 1 \end{bmatrix}$.
    \item Let $U = \begin{bmatrix} u_{11} & u_{12} \\ 0 & u_{22} \end{bmatrix}$.
    \item Since $F_{21} = l_{21} \cdot u_{11}$ and $F_{21} = 6$, $u_{11} = 4$, we get $l_{21} = \frac{6}{4} = \frac{3}{2}$.
    \item Solve for $U$ using the first row of $F$: $U = \begin{bmatrix} 4 & 3 \\ 0 & u_{22} \end{bmatrix}$. The second element of the second row gives $u_{22} = 3 - \frac{3}{2} \cdot 3 = -\frac{3}{2}$.
    \item The LU decomposition is $L = \begin{bmatrix} 1 & 0 \\ \frac{3}{2} & 1 \end{bmatrix}$, $U = \begin{bmatrix} 4 & 3 \\ 0 & -\frac{3}{2} \end{bmatrix}$.
  \end{enumerate}

  \columnbreak

  % === COLUMN 6: Quick Reference and Common Pitfalls ===
  \section*{Section 5}

  % Problem 11: Orthogonal Diagonalization
  \textbf{Problem 11: Orthogonal Diagonalization}

  Orthogonally diagonalize the matrix $F = \begin{bmatrix} 3 & -1 \\ -1 & 3 \end{bmatrix}$.

  \textbf{Solution:}
  \begin{enumerate}
    \item Find the eigenvalues by solving $\det(F - \lambda I) = 0$: $\lambda^2 - 6\lambda + 8 = 0$ gives $\lambda_1 = 2$ and $\lambda_2 = 4$.
    \item Find the eigenvectors: For $\lambda_1 = 2$, solve $(F - 2I)x = 0$ to get $x_1 = \begin{bmatrix} 1 \\ 1 \end{bmatrix}$ (after normalization). For $\lambda_2 = 4$, solve $(F - 4I)x = 0$ to get $x_2 = \begin{bmatrix} 1 \\ -1 \end{bmatrix}$ (after normalization).
    \item Construct $P = \begin{bmatrix} 1/\sqrt{2} & 1/\sqrt{2} \\ 1/\sqrt{2} & -1/\sqrt{2} \end{bmatrix}$ and verify $P^T F P = \begin{bmatrix} 2 & 0 \\ 0 & 4 \end{bmatrix}$.
  \end{enumerate}


  % Problem 12: Singular Value Decomposition
  \textbf{Problem 12: Singular Value Decomposition (SVD)}

  Find the singular value decomposition of the matrix $G = \begin{bmatrix} 3 & 0 \\ 0 & -2 \end{bmatrix}$.

  \textbf{Solution:}
  \begin{enumerate}
    \item Compute the eigenvalues of $G^TG$: Since $G^TG = \begin{bmatrix} 9 & 0 \\ 0 & 4 \end{bmatrix}$, the eigenvalues are $9$ and $4$.
    \item The singular values are the square roots of the eigenvalues: $\sigma_1 = 3$ and $\sigma_2 = 2$.
    \item The right singular vectors are the eigenvectors of $G^TG$, which correspond to $v_1 = \begin{bmatrix} 1 \\ 0 \end{bmatrix}$ and $v_2 = \begin{bmatrix} 0 \\ 1 \end{bmatrix}$.
    \item The left singular vectors are obtained by normalizing the columns of $GU$, where $U$ is the matrix of right singular vectors. In this case, they remain the same as $v_1$ and $v_2$.
    \item Assemble the SVD: $G = U\Sigma V^T$ with $U = V = \begin{bmatrix} 1 & 0 \\ 0 & 1 \end{bmatrix}$ and $\Sigma = \begin{bmatrix} 3 & 0 \\ 0 & 2 \end{bmatrix}$.
  \end{enumerate}




\end{multicols*}

\pagebreak

\end{document}