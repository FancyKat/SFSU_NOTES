\section*{Development for Fleet Management Dashboard}
\addcontentsline{toc}{section}{Development for Fleet Management Dashboard}

\subsection*{Setting Up the Development Environment}
Before starting the coding process, ensure that the development environment is fully prepared. This includes having Next.js and all necessary dependencies installed, as well as configuring any additional tools like linters or formatters.

\textbf{Initial Setup:}
\begin{itemize}
    \item Initialize the Next.js project with \texttt{npx create-next-app}.
    \item Set up version control using Git to track changes and collaborate efficiently.
\end{itemize}

\subsection*{Frontend Development}
The frontend development will focus on transforming the finalized design mockups into a functional and interactive user interface using Next.js and React.

\textbf{Key Tasks:}
\begin{itemize}
    \item Develop the UI components based on the design specifications, ensuring they are responsive and accessible.
    \item Implement state management to handle data across components.
    \item Integrate real-time data fetching mechanisms to display vehicle tracking and alerts.
\end{itemize}

\subsection*{Backend Integration (Optional)}
If the dashboard requires a custom backend:

\textbf{Development Tasks:}
\begin{itemize}
    \item Set up a Node.js (or another preferred runtime) environment for the backend.
    \item Develop RESTful APIs to handle requests between the frontend and the database.
    \item Implement authentication and authorization to secure data access.
\end{itemize}

\subsection*{Database Connection}
Establish a connection to the chosen database to store and retrieve the dashboard's data effectively.

\textbf{Implementation Steps:}
\begin{itemize}
    \item Choose the appropriate database drivers or ORMs for interacting with your database.
    \item Design the database schema to efficiently store fleet data, including vehicle locations, statuses, and maintenance records.
    \item Implement data access layers in the backend to interact with the database securely and efficiently.
\end{itemize}

\subsection*{Testing and Validation}
\begin{itemize}
    \item Regularly test the application for functionality, usability, and responsiveness.
    \item Implement unit and integration tests to ensure individual components and their interactions function as expected.
    \item Use end-to-end testing frameworks to simulate user scenarios and validate the complete workflow of the application.
\end{itemize}


\section*{Terminal Command Guide for Basic Next.js and TypeScript Setup}

This guide will walk you through the terminal commands necessary to create a basic Next.js project configured with TypeScript. By following these steps, you'll establish a foundational file structure for your Fleet Management Dashboard.

\subsection*{Creating Your Next.js Project}
Start by creating a new Next.js project. Open your terminal and execute the following command:

\begin{verbatim}
npx create-next-app@latest my-fleet-dashboard --typescript
\end{verbatim}

This command creates a new Next.js project with TypeScript support enabled by default.

\subsection*{Exploring the Initial Project Structure}
Navigate to your project directory to see the initial setup:

\begin{verbatim}
cd my-fleet-dashboard
\end{verbatim}

You should see a basic file structure created for you. Here's an overview of the primary files and directories:

\begin{itemize}
    \item \texttt{pages/} - Contains your application's pages. Next.js uses the files in this directory to create the routes automatically.
    \item \texttt{public/} - Used for static files like images, fonts, etc.
    \item \texttt{styles/} - Holds the CSS/SCSS files for styling your application.
    \item \texttt{node\_modules/} - Contains all the project's npm dependencies.
    \item \texttt{package.json} - Manages the project's metadata and dependencies.
    \item \texttt{tsconfig.json} - Configures TypeScript options for your project.
\end{itemize}

\subsection*{Running Your Development Server}
To start the development server and see your new project in action, run:

\begin{verbatim}
npm run dev
\end{verbatim}

Or if you prefer Yarn:

\begin{verbatim}
yarn dev
\end{verbatim}

This command starts the Next.js development server on \texttt{http://localhost:3000}. You can open this URL in a browser to view your project.

\subsection*{Final Project File Tree}
Once you're familiar with the basic structure, your project directory might look like this:

\begin{verbatim}
my-fleet-dashboard/
├── node_modules/
├── pages/
│   ├── _app.tsx
│   └── index.tsx
├── public/
├── styles/
│   ├── globals.css
│   └── Home.module.css
├── .gitignore
├── package.json
├── tsconfig.json
└── yarn.lock (or package-lock.json for npm)
\end{verbatim}

This file tree provides a visual representation of the basic project structure, helping you understand where to add or modify files as your project develops.
