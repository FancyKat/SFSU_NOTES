\documentclass[12pt]{article}
\usepackage{fancyhdr}
\usepackage{amsmath, amssymb, amsthm}

\newtheorem{proposition}{Proposition}[section]

% Define a new proposition environment with a custom numbering
\newenvironment{customprop}[1]{
  \renewcommand\theproposition{#1}
  \proposition
}{\endproposition}

\begin{document}
\pagestyle{fancy}
\fancyhf{} % clear all header and footer fields
\fancyhead[L]{{MATH 301GW} \\ Professor Matthias Beck}
\fancyhead[R]{Marty Martin \\ pmartin@sfsu.edu \\ \today}
\setlength{\headheight}{41.54604pt}
\addtolength{\topmargin}{-27.04604pt}

\begin{customprop}{2.18}
    This is Proposition 2.18.
\end{customprop}

\begin{proof}
    We assume that $a$ and $b$ are type 2 integers and will prove that $a \cdot b$ is a type 1 integer.  Since $a$ and $b$ are type 2 integers, there exist integers $m$ and $n$ such that
    \[
      a = 3m + 2 \text{ ~~~~~~and~~~~~ } b = 3n + 2.
    \]
    We can now  use substitution and algebra .........
    \begin{align*}
      ab  & = (3m + 2)(3n + 2)      \\
          & = 9mn + 6m + 6n + 4     \\
          & = 9mn + 6m + 6n + 3 + 1
    \end{align*}
\end{proof}
  

\begin{customprop}{2.21}
This is Proposition 2.21.
\end{customprop}

\end{document}
