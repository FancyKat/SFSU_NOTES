% Equality "="
\unnumberedsection{Equality $\boldsymbol{=}$}
The symbol $\boldsymbol{=}$ means \textbf{equals}. To say $m = n$ means that $m$ and $n$ are the same number. Some properties are:
\begin{enumerate}[label=\roman*.]
  \item $\mathit{m = m}$ \hfill (\textbf{reflexivity})
  \item If $m = n$ then $n = m$ \hfill (\textbf{symmetry})
  \item If $m = n$ and $n = p$ then $m = p$ \hfill (\textbf{transitivity})
  \item If $m = n$, then $n$ can be substituted for $m$ in \\ any statement without changing the meaning \hfill (\textbf{replacement})
\end{enumerate}


% Inequality "≠"
\unnumberedsection{Inequality $\boldsymbol{\neq}$}
The symbol $\boldsymbol{\neq}$ means \textbf{is not equal to}. To say $m \neq n$ means that $m$ and $n$ are different numbers. Note that $\boldsymbol{\neq}$ satisfies \textbf{symmetry}, but not \textbf{transitivity} and \textbf{reflexivity}.

% Continue with other definitions related to equality and inequality if necessary...

% In the set of "∈"
\unnumberedsection{In the set of $\boldsymbol{\in}$}
The symbol $\boldsymbol{\in}$ means \textbf{is an element of}. For example, $0 \in \mathbb{Z}$ means "0 is an element of the set $\mathbb{Z}$."

% Not in the set of "/∈"
\unnumberedsection{Not in the set of $\boldsymbol{\notin}$}
The symbol $\boldsymbol{\notin}$ means \textbf{is not an element of}. For example, $0.5 \notin \mathbb{Z}$ means "0.5 is not an element of the set $\mathbb{Z}$."

% Divisibility
\unnumberedsection{Divisibility}
When $m$ and $n$ are integers, we say $m$ is divisible by $n$ (or alternatively, $n$ divides $m$) if there exists $j \in \mathbb{Z}$ such that $m = jn$. We use the notation $n|m$.

% 2 and other integers
\unnumberedsection{2 and other integers}
$\boldsymbol{2}$ is defined as $2 = 1+1$ and $\boldsymbol{3}$ is $2+1$ and so on.

% Even Integers
\unnumberedsection{Even Integers}
Even integers are defined to be those integers that are divisible by $2$. That is, $x = 2j$, where $j \in \mathbb{Z}$.

% Subtraction
\unnumberedsection{Subtraction}
Subtraction is defined as $m - n$ is defined to be $m + (-n)$.

\section*{Number Theory}

% Power
\unnumberedsection{Power}
Let $b$ be a fixed integer. We define $b^k$ for all integers $k \geq 0$ by:
\begin{enumerate}
    \item $\boldsymbol{b^0 := 1}$
    \item Assuming $b^n$ is defined, let $\boldsymbol{b^{n+1} := b^n \cdot b}$
\end{enumerate}

% Sum
\unnumberedsection{Sum}
Let $(x_j)_{j=1}^{\infty}$ be a sequence of integers. $(x_j)_{j=1}^3 = \{1,2,3\}$. For each $k \in \mathbb{N}$, we want to define an integer called $\Sigma_{j=1}^k x_j$:
\begin{enumerate}[label=\textbf{\arabic*.}]
    \item Define $\mathbf{\Sigma_{j=1}^1 x_j}$ to be $x_1$
    \item Assuming $\Sigma_{j=1}^n x_j$ is already defined, we define $\mathbf{\Sigma_{j=1}^{n+1} x_j}$ to be $\Sigma_{j=1}^n x_j + x_{n+1}$
\end{enumerate}

% Product
\unnumberedsection{Product}
Let $(x_j)_{j=1}^{\infty}$ be a sequence of integers. $(x_j)_{j=1}^3 = \{1,2,3\}$. For each $k \in \mathbb{N}$, we want to define an integer called $\Pi_{j=1}^k x_j$:
\begin{enumerate}[label=\textbf{\arabic*.}]
    \item Define $\mathbf{\Pi_{j=1}^1 x_j}$ to be $x_1$
    \item Assuming $\Pi_{j=1}^n x_j$ is already defined, we define $\mathbf{\Pi_{j=1}^{n+1} x_j}$ to be $\Pi_{j=1}^n x_j \cdot x_{n+1}$
\end{enumerate}

% Non-negative integer (N with 0 included)
\unnumberedsection{Non-negative integer ($\mathbb{Z}_{\geq 0}$)}
$\mathbb{Z}_{\geq 0} := \{m \in \mathbb{Z} : m \geq 0\}$

% Factorial
\unnumberedsection{Factorial}
We define $k!$ ("$k$ factorial") for all integers $k \geq 0$ by:
\begin{enumerate}[label=\textbf{\arabic*.}]
    \item Define $\mathbf{0! := 1}$
    \item Assuming $n!$ is defined (where $n \in \mathbb{Z}_{\geq 0}$), define $\mathbf{(n+1)! := (n!) \cdot (n+1)}$
\end{enumerate}

% Subset
\unnumberedsection{Subset ($\subseteq$)}
$A \subseteq B$ means that if $x \in A$, then $x \in B$

% The Empty Set
\unnumberedsection{The Empty Set ($\emptyset$)}
The empty set is defined as a set that contains no elements.

% Equal Sets
\unnumberedsection{Equal Sets ($=$)}
The set $A$ is equal to $B$ means that $A \subseteq B$ and $B \subseteq A$. In order to prove two sets are equal, you have to complete two proofs.

% Functions
\unnumberedsection{Functions}
\subsection*{\textbf{Informal Definition}}
A function consists of:
\begin{itemize}
    \item a set $A$ called the \textbf{domain} of the function
    \item a set $B$ called the \textbf{codomain} of the function
    \item a rule $f$ that assigns to each $a \in A$ an element $f(a) \in B$. Shorthand for this is $f : A \rightarrow B$
\end{itemize}
\subsection*{\textbf{Abstract Definition}}
A function with domain $A$ and codomain $B$ is a subset of $\Gamma$ of $A \times B$ such that for each $a \in A$, there is one and only one element of $\Gamma$ whose first entry is $a$. If $(a,b) \in \Gamma$, we write $b = f(a)$.
