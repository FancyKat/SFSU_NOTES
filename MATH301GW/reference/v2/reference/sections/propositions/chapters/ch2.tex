% Proposition 2.2
\section*{Proposition 2.2:}
For every $m \in \mathbb{Z}$, one and only one of the following is true: $m \in \mathbb{N}$, $-m \in \mathbb{N}$, or $m = 0$.

% Proposition 2.3
\section*{Proposition 2.3:}
$1 \in \mathbb{N}$.

% Proposition 2.4
\section*{Proposition 2.4:}
Let $m,n,p \in \mathbb{Z}$. If $m < n$ and $n < p$, then $m < p$.

% Proposition 2.5
\section*{Proposition 2.5:}
For each $n \in \mathbb{N}$, there exists $m \in \mathbb{N}$ such that $m > n$.

% Proposition 2.6
\section*{Proposition 2.6:}
Let $m,n \in \mathbb{Z}$. If $m \leq n \leq m$, then $m = n$.

% Proposition 2.7
\section*{Proposition 2.7:}
\begin{enumerate}[label=(\roman*)]
    \item If $m < n$, then $m+p < n+p$.
    \item If $m < n$ and $p < q$, then $m+p < n+q$.
    \item If $0 < m < n$ and $0 < p \leq q$, then $mp < nq$.
    \item If $m < n$ and $p < 0$, then $np < mp$.
\end{enumerate}

% Proposition 2.8
\section*{Proposition 2.8:}
Let $m,n \in \mathbb{Z}$. Exactly one of the following is true: $m < n$, $m = n$, $m > n$.

% Proposition 2.9
\section*{Proposition 2.9:}
Let $m \in \mathbb{Z}$. If $m \neq 0$ then $m^2 \in \mathbb{N}$.

% Proposition 2.10
\section*{Proposition 2.10:}
The equation $x^2 = -1$ has no solution in $\mathbb{Z}$.

% Proposition 2.11
\section*{Proposition 2.11:}
Let $m \in \mathbb{N}$ and $n \in \mathbb{Z}$. If $mn \in \mathbb{N}$, then $n \in \mathbb{N}$.

% Proposition 2.12
\section*{Proposition 2.12:}
For all $m,n,p \in \mathbb{Z}$:
\begin{enumerate}[label=(\roman*)]
    \item $-m < -n$ if and only if $m > n$.
    \item If $p > 0$ and $mp < np$ then $m < n$.
    \item If $p < 0$ and $mp < np$ then $n < m$.
    \item If $m \leq m$ and $0 \leq p$ then $mp \leq np$.
\end{enumerate}


% Proposition 2.13
\section*{Proposition 2.13:}
$\mathbb{N} = \{ n \in \mathbb{Z} : n > 0 \}$.

% Proposition 2.14
\section*{Proposition 2.14:}
\begin{enumerate}[label=(\roman*)]
    \item $1 \in \mathbb{N}$.
    \item If $n \in \mathbb{N}$, then $n+1 \in \mathbb{N}$.
\end{enumerate}

% Axiom 2.15 (Induction Axiom)
\section*{Axiom 2.15 (Induction Axiom):}
If a subset $A \subseteq \mathbb{Z}$ satisfies:
\begin{enumerate}
    \item $1 \in A$, and
    \item If $n \in A$, then $n+1 \in A$,
\end{enumerate}
then $\mathbb{N} \subseteq A$.

% Proposition 2.16
\section*{Proposition 2.16:}
Let $B \subseteq \mathbb{Z}$ be such that:
\begin{enumerate}
    \item $1 \in B$, and
    \item If $n \in B$, then $n+1 \in B$,
\end{enumerate}
then $B = \mathbb{N}$.

% Theorem 2.17 (Principle of mathematical induction-first form)
\section*{Theorem 2.17 (Principle of Mathematical Induction - First Form):}
Let $P(k)$ be a statement depending on a variable $k \in \mathbb{N}$. In order to prove the statement "P(k) is true for all $k \in \mathbb{N}$," it is sufficient to prove:
\begin{enumerate}
    \item $P(1)$ is true, and
    \item For any given $n \in \mathbb{N}$, if $P(n)$ is true, then $P(n+1)$ is true.
\end{enumerate}

% Proposition 2.18
\section*{Proposition 2.18:}
\begin{enumerate}[label=(\roman*)]
    \item For all $k \in \mathbb{N}$, $k^3+2k$ is divisible by $3$.
    \item For all $k \in \mathbb{N}$, $k^4-6k^3+11k^2-6k$ is divisible by $4$.
    \item For all $k \in \mathbb{N}$, $k^3+5k$ is divisible by $6$.
\end{enumerate}

% Proposition 2.20
\section*{Proposition 2.20:}
For all $k \in \mathbb{N}$, $k \geq 1$.

% Proposition 2.21
\section*{Proposition 2.21:}
There exists no integer $x$ such that $0 < x < 1$.

% Corollary 2.22
\section*{Corollary 2.22:}
Let $n \in \mathbb{Z}$. There exists no integer $x$ such that $n < x < n+1$.

% Proposition 2.23
\section*{Proposition 2.23:}
Let $m,n \in \mathbb{N}$. If $n$ is divisible by $m$, then $m \leq n$.

% Proposition 2.24
\section*{Proposition 2.24:}
For all $k \in \mathbb{N}$, $k^2 + 1 > k$.

% Proposition 2.26
\section*{Proposition 2.26:}
For all integers $k \geq -3$, $3k^2 + 21k + 37 \geq 0$.

% Proposition 2.27
\section*{Proposition 2.27:}
For all integers $k \geq 2$, $k^2 < k^3$.

% Theorem 2.32
\section*{Theorem 2.32 (Well-Ordering Principle):}
Every nonempty subset of $\mathbb{N}$ has a smallest element.

% Proposition 2.33
\section*{Proposition 2.33:}
Let $A$ be a nonempty subset of $\mathbb{Z}$ and $b \in \mathbb{Z}$, such that for each $a \in A$, $b \leq a$. Then $A$ has a smallest element.

% Proposition 2.34
\section*{Proposition 2.34:}
If $m$ and $n$ are integers that are not both $0$, then
\[ S = \{ k \in \mathbb{N} : k = mx + ny \text{ for some } x,y \in \mathbb{Z} \} \]


% Theorem 4.4
\section*{Theorem 4.4:}
A legitimate method of describing a sequence $(y_j)_{j=m}^{\infty}$ is:
\begin{enumerate}
    \item to name $y_m$, and
    \item to state a formula describing $y_{n+1}$ in terms of $y_n$, for each $n \geq m$.
\end{enumerate}

% Proposition 4.5
\section*{Proposition 4.5:}
For all $k \in \mathbb{Z}_{\geq 0}$, $k! \in \mathbb{N}$.

% Proposition 4.6
\section*{Proposition 4.6:}
Let $b \in \mathbb{Z}$ and $k,m \in \mathbb{Z}_{\geq 0}$.
\begin{enumerate}
    \item if $b \in \mathbb{N}$ then $b^k \in \mathbb{N}$
    \item $b^mb^k = b^{m+k}$
    \item $(b^m)^k = b^{mk}$
\end{enumerate}

% Proposition 4.7
\section*{Proposition 4.7:}
For all $k \in \mathbb{N}$:
\begin{enumerate}
    \item $5^{2k-1}$ is divisible by $24$
    \item $2^{2k+1} + 1$ is divisible by $3$
    \item $10^{k+3} \cdot 4^{k+2} + 5$ is divisible by $9$
\end{enumerate}

% Proposition 4.8
\section*{Proposition 4.8:}
For all $k \in \mathbb{N}$, $4k > k$.

% Proposition 4.11
\section*{Proposition 4.11:}
Let $k \in \mathbb{N}$:
\begin{enumerate}
    \item $\sum_{j=1}^k j = \frac{k(k+1)}{2}$
    \item $\sum_{j=1}^k j^2 = \frac{k(k+1)(2k+1)}{6}$
\end{enumerate}

% Proposition 4.13
\section*{Proposition 4.13:}
For $x \neq 1$ and $k \in \mathbb{Z}_{\geq 0}$,
\[ \sum_{j=0}^k x^j = \frac{1 - x^{k+1}}{1 - x} \]

% Proposition 4.15
\section*{Proposition 4.15:}
\begin{enumerate}
    \item Let $m \in \mathbb{Z}$ and let $(x_j)_{j=1}^{\infty}$ be a sequence in $\mathbb{Z}$. Then for all $k \in \mathbb{N}$:
    \[ m \cdot \sum_{j=1}^k x_j! = \sum_{j=1}^k (mx_j) \]
    \item If $x_j = 1$ for all $j \in \mathbb{N}$, then for all $k \in \mathbb{N}$:
    \[ \sum_{j=1}^k x_j = k \]
    \item If $x_j = n \in \mathbb{Z}$ for all $j \in \mathbb{N}$, then for all $k \in \mathbb{N}$:
    \[ \sum_{j=1}^k x_j = kn \]
\end{enumerate}


% Proposition 4.16
\section*{Proposition 4.16:}
Let $(x_j)_{j=1}^{\infty}$ and $(y_j)_{j=1}^{\infty}$ be sequences in $\mathbb{Z}$, and let $a,b,c \in \mathbb{Z}$ be such that $a \leq b < c$.
\begin{enumerate}
    \item $\displaystyle\sum_{j=a}^{c} x_j = \displaystyle\sum_{j=a}^{b} x_j + \displaystyle\sum_{j=b+1}^{c} x_j$
    \item $\displaystyle\sum_{j=a}^{b} (x_j + y_j) = \displaystyle\sum_{j=a}^{b} x_j! + \displaystyle\sum_{j=a}^{b} y_j!$
\end{enumerate}

% Proposition 4.17
\section*{Proposition 4.17:}
Let $(x_j)_{j=1}^{\infty}$ be a sequence in $\mathbb{Z}$, and let $a,b, r \in \mathbb{Z}$ be such that $a \leq b$. Then
$\displaystyle\sum_{j=a}^{b} x_j = \displaystyle\sum_{j=a}^{b+r} x_j - r$

% Proposition 4.18
\section*{Proposition 4.18:}
Let $(x_j)_{j=1}^{\infty}$ and $(y_j)_{j=1}^{\infty}$ be sequences in $\mathbb{Z}$ such that $x_j \leq y_j$ for all $j \in \mathbb{N}$. Then for all $k \in \mathbb{N}$,
$\displaystyle\sum_{j=1}^{k} x_j \leq \displaystyle\sum_{j=1}^{k} y_j$

% Theorem 4.19
\section*{Theorem 4.19:}
Let $k,m \in \mathbb{Z}_{\geq 0}$, where $m \leq k$. Then $m!(k-m)!$ divides $k!$.

% Corollary 4.20
\section*{Corollary 4.20:}
For $1 \leq m \leq k$,
$\binom{k+1}{m} = \binom{k}{m-1} + \binom{k}{m}$

% Theorem 4.21
\section*{Theorem 4.21 (Binomial theorem for integers):}
If $a,b \in \mathbb{Z}$ and $k \in \mathbb{Z}_{\geq 0}$ then
$(a+b)^k = \displaystyle\sum_{m=0}^{k} \binom{k}{m} a^{k-m} b^m$

% Corollary 4.22
\section*{Corollary 4.22:}
For $k \in \mathbb{Z}_{\geq 0}$,
$\displaystyle\sum_{m=0}^{k} \binom{k}{m} = 2^k$

% Theorem 4.24
\section*{Theorem 4.24 (Principle of mathematical induction —second form):}
Let $P(k)$ be a statement depending on a variable $k \in \mathbb{N}$. In order to prove the statement ”$P(k)$ is true for all $k \in \mathbb{N}$” it is sufficient to prove:
\begin{enumerate}
    \item $P(1)$ is true and
    \item if $P(j)$ is true for all integers $j$ such that $1 \leq j \leq n$, then $P(n+1)$ is true
\end{enumerate}

% Proposition 4.29
\section*{Proposition 4.29:}
The $k$th Fibonacci number is given directly by the formula
$f_k = \frac{1}{\sqrt{5}} \left(\left(\frac{1 + \sqrt{5}}{2}\right)^k - \left(\frac{1 - \sqrt{5}}{2}\right)^k\right)$


% Proposition 4.30
\section*{Proposition 4.30:}
For all $k,m \in \mathbb{N}$, where $m \geq 2$,
\[ f_{m+k} = f_{m-1} f_k + f_m f_{k+1} \]

% Proposition 4.31
\section*{Proposition 4.31:}
For all $k \in \mathbb{N}$,
\[ f_{2k+1} = f_k^2 + f_{k+1}^2 \]

% Proposition 4.32
\section*{Proposition 4.32:}
For all $k,m \in \mathbb{N}$, $f_{mk}$ is divisible by $f_m$.

% Chapter 5
\section*{Chapter 5}

% Proposition 5.1
\section*{Proposition 5.1: Let $A,B,C$ be sets:}
\begin{enumerate}
    \item $A \subseteq A$
    \item If $A \subseteq B$ and $B \subseteq C$ then $A \subseteq C$
\end{enumerate}

% Proposition 5.2
\section*{Proposition 5.2:}
\[ \{7m+1 : m \in \mathbb{Z}\} = \{7n-6 : n \in \mathbb{Z}\} \]

% Proposition 5.4
\section*{Proposition 5.4: Let $A,B,C$ be sets:}
\begin{enumerate}
    \item $A = A$
    \item if $A = B$ then $B = A$
    \item If $A = B$ and $B = C$ then $A = C$
\end{enumerate}

% Proposition 5.6
\section*{Proposition 5.6:}
If the sets $\emptyset_1$ and $\emptyset_2$ have the property that $x \in \emptyset_1$ is never true and $x \in \emptyset_2$ is never true, then $\emptyset_1 = \emptyset_2$.

% Proposition 5.7
\section*{Proposition 5.7:}
The empty set is a subset of every set, that is, for every set $S$, $\emptyset \subseteq S$.

% Proposition 5.14
\section*{Proposition 5.14: Let $A,B \subseteq X$:}
$A \subseteq B$ if and only if $B^c \subseteq A^c$.

% Theorem 5.15 (De Morgan’s laws)
\section*{Theorem 5.15 (De Morgan’s laws):}
Given two subsets $A,B \subseteq X$,
\[ (A \cap B)^c = A^c \cup B^c \quad \text{and} \quad (A \cup B)^c = A^c \cap B^c \]

% Proposition 5.20
\section*{Proposition 5.20: Let $A,B,C$ be sets:}
\begin{enumerate}
    \item $A \times (B \cup C) = (A \times B) \cup (A \times C)$
    \item $A \times (B \cap C) = (A \times B) \cap (A \times C)$
\end{enumerate}
